\chapter{Differential Cross Section Derivation}
%---------------------------------------------------------------------
\section{Probability Calculation}
\label{full}

\subsection{Differential Cross Section at the Parton Level}

The matrix element analysis technique reconstructs each event to the
final state four-vectors to evaluate the signal and background leading
order matrix element. The following sections derive the signal and
background probabilities starting from the final state at the parton
level and then relating these objects to the physical quantities
measured in the detector. The following also assumes a lepton,
neutrino, and two quarks in the final state.

The probability density for a process to occur at a hadron-hadron
collider is given as an integral of the hard-scatter differential
cross section over all possible ways of producing the process from the
quarks and gluons inside the hadron. This probability density, shown
below, is a convolution of the hard scatter differential cross section
with a parton distribution function for each of the two partons from
the hadrons with an integral over all possible momentum fractions
$x_{i}, x_{j}$ from each initial parton.
\begin{equation}
{\cal P}(\vec{y}) = \frac{1}{\sigma} \sum_{i,j}
\int f_{i}(q_{1}, Q^{2})dq_{1}
\times f_{j}(q_{2}, Q^{2})dq_{2}
\times d\sigma_{hs,ij}(\vec{y})
\end{equation}
\noindent where the normalization constant $\sigma$ is defined as
integral of the differential cross section over the initial- and
final-state phase spac:
\begin{equation}
\label{ds}
\sigma = \int  \sum_{i,j}
\int f_{i}(q_{1}, Q^{2})dq_{1}
\times f_{j}(q_{2}, Q^{2})dq_{2}
\times \pderiv{\sigma_{hs,ij}(\vec{y})}{\vec{y}} d\vec{y}
\end{equation}
\noindent and finally, the hard-scatter differential cross section is
defined as the product of the final state phase space factor, the
square of the matrix element amplitude and an overall flux factor:
\begin{equation}
d\sigma_{hs} = \frac{(2\pi)^4}{4}
\frac{{|\cal M|}^{2}}
{\sqrt{(q_{1}q_{2})^2 - m_{1}^2 m_{2}^2}}
\frac{d^{3}p_{1}}{(2\pi)^3 2E_{1}}
\frac{d^{3}p_{2}}{(2\pi)^3 2E_{2}}
\frac{d^{3}p_{\ell}}{(2\pi)^3 2E_{\ell}}
\frac{d^{3}p_{\nu}}{(2\pi)^3 2E_{\nu}}
\delta^{4}(q_{1}q_{2};p_{1},p_{2},p_{\ell},p_{\nu})
\end{equation}


\subsection{Evaluating the Hard Scatter Differential Cross Section}

The following section evaluates the differential cross section shown
in Eq.~\ref{ds} given a set of inital and final state four-vectors.

The first assumption made is that all collisions occur along the beam
axis with no net transverse momentum. This means the initial state
four vectors can be written as
\begin{equation}
q_{1} = ( E_{beam} x_{1}, 0, 0, E_{beam} x_{1} )
\end{equation}
\begin{equation}
q_{2} = ( E_{beam} x_{2}, 0, 0, -E_{beam} x_{2} )
\end{equation}
\noindent The next assumption is that all particle masses are known
and are negligible compared to their energies and thus can be ignored
for this calculation. The flux factor (shown below) in the hard
scatter cross section can now be written in terms in the two momentum
fractions of the incoming partons:
\begin{equation}
\frac{1}{\sqrt{(q_{1}q_{2})^2 - m_{1}^2m_{2}^2}} \rightarrow
\frac{1}{\sqrt{(q_{1}q_{2})^2}} \rightarrow
\frac{1}{2E_{beam}x_{1}x_{2}}
\end{equation}
\noindent For the remainder of the note, the following notation will
be used to distinquish quarks, leptons, and neutrinos: $p_{\ell}$ is
the momemtum of the lepton, $p_{1,2}$ is the momentum of the first and
second final state partons, and $p_{\nu}$ is the neutrino
momentum. Because the phase space is written in terms of rectangular
coordinates, the next step towards the final differential cross
section equation is to redefine the phase space factors in terms of
spherical coordiniates. This is done for all final state particles
except the neutrino for reasons that will be clear later in the
document.
\begin{eqnarray}
d\Phi_{4} = 
\frac{d^{3}p_{1}}{(2\pi)^3 2E_{1}}
\frac{d^{3}p_{2}}{(2\pi)^3 2E_{2}}
\frac{d^{3}p_{\ell}}{(2\pi)^3 2E_{\ell}}
\frac{d^{3}p_{\nu}}{(2\pi)^3 2E_{\nu}} \delta^{4} 
\rightarrow \\
\frac{1}{16(2\pi)^{12}}
\frac{|p_{1}|^{2}d|p_{1}|d\Omega_{1}}{E_{1}}
\frac{|p_{2}|^{2}d|p_{2}|d\Omega_{2}}{E_{2}}
\frac{|p_{\ell}|^{2}d|p_{\ell}|d\Omega_{\ell}}{E_{\ell}}
\frac{d^{x}_{\nu} d^{y}_{\nu} d^{z}_{\nu}}{E_{\nu}}\delta^{4} 
%\frac{d^{3}p_{1}}{(2\pi)^3 2E_{1}} \frac{d^{3}p_{2}}{(2\pi)^32E_{2}} \frac{d^{3}p_{\ell}}{(2\pi)^3 2E_{\ell}}
%\frac{d^{3}p_{\nu}}{(2\pi)^3 2E_{\nu}} \delta^{4} \righarrow 
%\frac{1}{16(2\pi)^{12}}
\end{eqnarray}
\noindent To summarize, the full hard scatter differential cross
section is now
\begin{equation}
\label{dhs}
d\sigma_{hs} = \frac{1}{128(2\pi)^{8}E_{beam}}
\frac{{|\cal M|}^{2}}{2x_{1}x_{2}}
\frac{|p_{1}|^{2}d|p_{1}|d\Omega_{1}}{E_{1}}
\frac{|p_{2}|^{2}d|p_{2}|d\Omega_{2}}{E_{2}}
\frac{|p_{\ell}|^{2}d|p_{\ell}|d\Omega_{\ell}}{E_{\ell}}
\frac{d^{x}_{\nu} d^{y}_{\nu} d^{z}_{\nu}}{E_{\nu}} \delta^{4} 
\end{equation} 


\subsection{Evaluating the Hadron-Hadron Differential Cross Section}

The next step to writing the full hadron-hadron differential cross
section is to rewrite Eq.~\ref{dhs} such that any integration over the
phase space will remove the four-dimensional delta function required
for energy and momentum conservation. The delta function is currently
written such that it will vanish only over integrations of total
$p_{x}, p_{y}, p_{z}$, and $E$. Because there was an original
assumption of no net transverse momentum in the collision, the total
$p_{x}$ and $p_{y}$ can be solved for the neutrino transverse
momentum.
\begin{eqnarray}
\sum_{i}P^{x}_{i} =
p^{x}_{1} + p^{x}_{2} + p^{x}_{\ell} + p^{x}_{\nu} = 0
\rightarrow p^{x}_{\nu} =
- p^{x}_{1} - p^{x}_{2} - p^{x}_{\ell} \\
\sum_{i}P^{y}_{i} =
p^{y}_{1} + p^{y}_{2} + p^{y}_{\ell} + p^{y}_{\nu} = 0
\rightarrow p^{y}_{\nu} =
- p^{y}_{1} - p^{y}_{2} - p^{y}_{\ell}
\end{eqnarray}
\noindent The total $p_{z}$ and requirement can be rewritten in terms
of the intial parton's momemtum fraction and the other final state
partons' $z$ momenta and thus solve for the neutrino $p_{z}$:
\begin{eqnarray}
\nonumber
\sum_{i}P^{z}_{i} =
p^{z}_{1} + p^{z}_{2} + p^{z}_{\ell} + p^{z}_{\nu} -
E_{beam}x_{1} + E_{beam}x_{2} = 0 \rightarrow \\
p^{z}_{\nu} =
- E_{beam}(x_{1} - x_{2}) - p^{z}_{1} - p^{z}_{2} - p^{z}_{\ell}
\end{eqnarray}
\noindent Finally, the total energy delta function implies the
following:
\begin{equation}
E_{beam}x_{1} + E_{beam}x_{2} = E_{1} + E_{2} + E_{\ell} + E_{\nu}
\end{equation}

At this point, is it useful to rewrite the full differential cross
section at the parton level:
\begin{eqnarray}
\nonumber
d\sigma(\vec{y}) = \sum_{i,j} \int f_{i}(x_{1}, Q^{2})dx_{1}
\times f_{j}(x_{2}, Q^{2})dx_{2}
\times \frac{1}{128(2\pi)^{8}E_{beam}}
\frac{{|\cal M|}^{2}}{2x_{1}x_{2}} \times \\
\nonumber
\frac{|p_{1}|^{2}d|p_{1}|d\Omega_{1}}{E_{1}}
\frac{|p_{2}|^{2}d|p_{2}|d\Omega_{2}}{E_{2}}
\frac{|p_{\ell}|^{2}d|p_{\ell}|d\Omega_{\ell}}{E_{\ell}}
\frac{d^{x}_{\nu} d^{y}_{\nu} d^{z}_{\nu}}{E_{\nu}} \times \\
\nonumber
\delta(p^{x}_{\nu} + p^{x}_{1} + p^{x}_{2} + p^{x}_{\ell}) \times \\
\nonumber
\delta(p^{y}_{\nu} + p^{y}_{1} + p^{y}_{2} + p^{y}_{\ell}) \times \\
\nonumber
\delta(p^{z}_{\nu} + E_{beam}(x_{1} - x_{2}) + p^{z}_{1}
+ p^{z}_{2} + p^{z}_{\ell}) \times \\
\delta(E_{beam}x_{1} + E_{beam}x_{2} - E_{1} - E_{2}
- E_{\ell} - E_{\nu})
\end{eqnarray}

The next step is to rewrite the integrational variables, $x_{1}$ and
$x_{2}$, in terms of the total energy and total $p_{z}$:
\begin{eqnarray}
\label{x1}
x_{1} = \frac{E_{tot} + p^{z}_{tot}}{2E_{beam}} \\
\label{x2}
x_{2} = \frac{E_{tot} - p^{z}_{tot}}{2E_{beam}}
\end{eqnarray}
\noindent Now, the integration over $x_{1}$ and $x_{2}$ can be
rewritten in terms of $E_{tot}$ and $p_{z}$:
\begin{eqnarray}
dx_{1}dx_{2} =
\frac{1}{J(x_{1},x_{2};E_{tot},p^{z}_{tot})}dE_{tot}dp^{z}_{tot} \\
J(x_{1},x_{2};E_{tot},p^{z}_{tot}) = 2E^{2}_{beam}
\end{eqnarray}
\noindent At this point the integration over the total energy and
$p_{z}$ will constrain the two incoming partons' momentum fractions
through Eq.~\ref{x1} and \ref{x2}.

The full differential cross section at the parton level can now be
written as
\begin{eqnarray}
\nonumber
d\sigma(\vec{y}) = \sum_{i,j} \int f_{i}(x_{1}, Q^{2})
\times 
f_{j}(x_{2}, Q^{2}) \times \frac{1}{128(2\pi)^{8}E_{beam}}
\frac{{|\cal M|}^{2}}{2x_{1}x_{2}} \times \\
\nonumber
\frac{|p_{1}|^{2}d|p_{1}|d\Omega_{1}}{E_{1}}
\frac{|p_{2}|^{2}d|p_{2}|d\Omega_{2}}{E_{2}}
\frac{|p_{\ell}|^{2}d|p_{\ell}|d\Omega_{\ell}}{E_{\ell}}
\frac{d^{x}_{\nu} d^{y}_{\nu} d^{z}_{\nu}}{E_{\nu}} \times \\
\int \frac{1}{2E^{2}_{beam}} dp^{z}_{tot}
\end{eqnarray}
\noindent where the implicit integration over the four dimensional
delta function yields the following formulas for the neutrino four
vector and the incoming partons' momentum fraction in terms of the
remaining differential variables.
\begin{eqnarray}
p^{x}_{\nu} = - p^{x}_{1} - p^{x}_{2} - p^{x}_{\ell} \\
p^{y}_{\nu} = - p^{y}_{1} - p^{y}_{2} - p^{y}_{\ell} \\
p^{z}_{\nu} = - p^{z}_{tot} - p^{z}_{1} - p^{z}_{2} - p^{z}_{\ell} \\
x_{1} = \frac{E_{1} + E_{2} + E_{\ell} + E_{\nu} + p^{z}_{tot}}
{2E_{beam}} \\
x_{2} = \frac{E_{1} + E_{2} + E_{\ell} + E_{\nu} - p^{z}_{tot}}
{2E_{beam}}
\end{eqnarray}


\subsection{Relating Reconstructed Objects to Partons}

The previous sections have calculated the differential cross section
for a hadron-hadron collision producing a lepton, neutrino, and two
partons in the final state. These particles are not exactly what is
measured in the detector and thus it is necessary to relate
quantities. To do this, the differential cross section is convoluted
with a function, $W(\vec{x}, \vec{y})$, which is the probability of
producing a final state, $\vec{y}$, and observed state, $\vec{x}$, in
the detector. The resulting differential cross section is then
integrated over the final state phase space, $d\vec{y}$:
\begin{equation}
\pderiv{\sigma^{'}(\vec{x})}{\vec{x}} =
\int \pderiv{\sigma(\vec{y})}{\vec{y}} W(\vec{x}, \vec{y}) d\vec{y}
\end{equation}
\noindent where the function $W(\vec{x}, \vec{y})$ is assumed to be
factorizable for each measured object:
\begin{equation}
W(\vec{x}, \vec{y}) = \prod_{i} W_{i}(\vec{x_{i}}, \vec{y_{i}})
\end{equation}


\subsubsection{Jets}

The transfer function for jets measured in the calorimter is assumed
to only be a function of the relative energy difference between the
two objects and all angles are assumed to be well measured:
\begin{equation}
W_{jet}(\vec{x}_{jet}, \vec{y}_{parton}) = W(E_{jet} -
E_{parton}) \times \delta(\Omega_{jet} - \Omega_{parton})
\end{equation}
\noindent where $W(E_{jet} - E_{parton})$ is parametrized using the
following functional form:
\begin{equation}
W(E_{jet} - E_{parton}) = \frac{e^{-\frac{(E_{jet}
- E_{parton} - p_{1})^{2}}{2p^{2}_{2}}} + p_{3}e^{-\frac{(E_{jet}
- E_{parton} - p_{4})^{2}}{2p^{5}_{2}}}}{2\pi(p_{2} + p_{3}p_{5})}
\end{equation}
\noindent where $p_{i} = \alpha_{i} + \beta_{i} \times E_{parton}$.
The five $\alpha$ and five $\beta$ parameters are determined by
minimizing a likelihood formed by measuring the parton energy in Monte
Carlo and the matched jet energy also in Monte Carlo. The parameters
used for this analysis were determined in several regions of the
calorimeter to account for the resolution differences in the detector.


\subsubsection{Electrons}

The transfer function for electrons is assumed to be
solely a function of the reconstructed energy of the electron,
$E_{reco}$, the parton energy of the electron, $E_{parton}$, and
$\theta$, the production angle with respect to the beam axis:
\begin{equation}
W_{electron}(\vec{x}_{reco}, \vec{y}_{parton}) = W(E_{reco},
E_{parton}, \theta) \times \delta(\Omega_{reco} - \Omega_{parton})
\end{equation}
\noindent where $W(E_{reco}, E_{parton}, \theta)$ is parametrized
using the following functional form:
\begin{eqnarray}
W(E_{reco}, E_{parton}, \theta) & = &
\frac{1}{2\pi\sigma}\mathrm{exp}
[-\frac{(E_{reco} - E_{center})^{2}}{2\sigma^{2}}]\\
E_{center} & = &
1.0002 E_{parton} + 0.324\,{GeV}/c^2 \\
\sigma & = &
0.028 E_{center} \oplus \textrm{Sampling}(E_{center},
\eta) E_{center} \oplus 0.4 \\
\textrm{Sampling}(E, \theta) & = &
\left[\frac{0.164}{\sqrt{E}} + \frac{0.122}{E}\right]
\textrm{exp}\left[\frac{\textrm{p1}(E)}
{\textrm{sin}\theta}-\textrm{p1}(E)\right] \\
\textrm{p1}(E)& = & 1.35193 - \frac{2.09564}{E} - \frac{6.98578}{E^2}.
\end{eqnarray}


\subsubsection{Muons}

The transfer function for muons is assumed to be a
function of
\begin{equation}
\Delta\left( \frac{q}{p_t}\right) =
\left( \frac{q}{p_t}\right)_{reco} -
\left( \frac{q}{p_t}\right)_{parton}
\end{equation}
\noindent and of $\eta_\mathrm{CFT}$,
\begin{equation}
W_{muon}(\vec{x}_{reco}, \vec{y}_{parton}) =
W\left(\Delta\left( \frac{q}{p_t}\right),
\eta_\mathrm{CFT}\right) \times 
\delta(\Omega_{reco} - \Omega_{parton})
\end{equation}
\noindent where $W\left(\Delta\left( \frac{q}{p_t}\right),
\eta_\mathrm{CFT}\right)$ is parametrized using a single Gaussian:
\begin{equation}
W\left(\Delta\left( \frac{q}{p_t}\right), \eta_\mathrm{CFT}\right) = 
\frac{1}{2\pi\sigma}\mathrm{exp}
\left\{-\frac{\left[\Delta
\left( \frac{q}{p_t} \right)\right]^2}
{2\sigma^2}\right\}
\end{equation}
\begin{equation}
\sigma  =  \left\{ 
\begin{array} {c@{\quad:\quad}l} \sigma_o &
|\eta_\mathrm{CFT}| \le \eta_o \\
\sqrt{\sigma^2_o + [c(|\eta_\mathrm{CFT}| - \eta_o)]^2} &
|\eta_\mathrm{CFT}| > \eta_o 
\end{array} \right.
\end{equation}

\noindent There are three fitted parameters in the above equations:
$\sigma_o$, $c$, and $\eta_o$, each of which is actually fitted by two
sub-parameters:
\begin{equation}
par = par(0) + par(1) \cdot 1/p_t.
\end{equation}
\noindent Furthermore, these parameters are derived for four classes
of events: those that were from before or after the 2004 shutdown,
when the magnetic field strength changed, and in each run range, those
that have an SMT hit and those that do not.

As a simplification, we assume $q_{reco} = q_{parton}$, that is, we
do not consider charge misidentification


\subsection{Full Differential Cross Section and Normalization}

The full differential cross section at the detector object level can
now be written as
\begin{eqnarray}
\label{fullds}
\nonumber
\pderiv{\sigma^{'}(\vec{x})}{\vec{x}} =
\int dp^{z}_{tot}dq_{1}dq_{2}dp_{\ell} \sum_{i,j}
f_{i}(q_{1}, Q^{2}) \times f_{j}(q_{2}, Q^{2}) \\
\times \frac{1}{256(2\pi)^{8}E^{3}_{beam}}
\frac{{|\cal M|}^{2}}{2x_{1}x_{2}} \times
\frac{p_{1}^{2}}{E_{1}} \frac{p_{2}^{2}}{E_{2}} 
\frac{p_{\ell}^{2}}{E_{\ell}} \frac{1}{E_{\nu}}
\times W_{Lepton}W_{Jet1}W_{Jet2}
\end{eqnarray}
\noindent The final step to evaluating the probability density is to
properly normalize the differential cross section in
Eq.~\ref{fullds}. This is done by integration of the differential
cross section over all possible states in the detector. Since the
event selection cuts will change the number events due to acceptance
losses, this must be accounted for in the overall normalization (cross
section) calculation. The total cross section is then written as
\begin{eqnarray}
\label{norm}
\nonumber
\sigma = \int \pderiv{\sigma^{'}(\vec{x})}{\vec{x}} d\vec{x} =
\int d\vec{x}dp^{z}_{tot}dq_{1}dq_{2}dp_{\ell} \sum_{i,j}
f_{i}(q_{1}, Q^{2}) \times f_{j}(q_{2}, Q^{2}) \\
\times \frac{1}{256(2\pi)^{8}E^{3}_{beam}}
\frac{{|\cal M|}^{2}}{2x_{1}x_{2}} \times
\frac{p_{1}^{2}}{E_{1}} \frac{p_{2}^{2}}{E_{2}} 
\frac{p_{\ell}^{2}}{E_{\ell}} \frac{1}{E_{\nu}}
\times W_{Lepton}W_{Jet1}W_{Jet2} \times \Theta_{\rm{Cuts}}(\vec{x})
\end{eqnarray}

 \section{Integration Variable Remapping}

\subsection{Introduction}
This section will layout the jacobian needed for the 10 $\rar$ 10 remapping of
variables for the parton level cross section. The base variables used are shown below.

\begin{itemize}
\item $p_{3}$: Absolute momentum of the lepton
\item $p_{5}$: Absolute momentum of the first quark
\item $p_{6}$: Absolute momentum of the second quark
\item $p_{tot}^{z}$: Total $p_{z}$ of the system
\item $cos(\theta_{3})$: Cosine($\theta$) of the lepton
\item $\phi_{3}$: $\phi$ of the lepton
\item $cos(\theta_{5})$: Cosine($\theta$) of the first quark
\item $\phi_{5}$: $\phi$ of the first quark
\item $cos(\theta_{6})$: Cosine($\theta$) of the second quark
\item $\phi_{6}$: $\phi$ of the second quark
\end{itemize}

Other variables that are useful for the integration are
\begin{itemize}
\item $m_{34}$: Mass of the lepton and neutrino (W mass)
\item $m_{345}$: Mass of the lepton, neutrino, and first quark (top mass)
\item $m_{56}$: Mass of the first and second quark ($b\bar{b}$ mass)
\end{itemize}

Since some of these variables are sharp peaks (W and top masses), it is much
better to sample from the expected distribution rather than make requirements of
the invariant masses. The W and top masses are expected to follow a Breit-Wigner
distribution shown below.

\begin{equation}
\sigma(M_{34}) = \frac{1}{\pi} \left[ \frac{\gamma}{(M_{34} - M_{W})^{2} +
\gamma^{2}} \right]
\end{equation}

where $M_{34}$ is the mass of the lepton and neutrino. Similarly, the top mass
has the following expected distribution.

\begin{equation}
\sigma(M_{345}) = \frac{1}{\pi} \left[ \frac{\gamma}{(M_{345} - M_{top})^{2} +
\Gamma^{2}} \right]
\end{equation}

where  $M_{345}$ is the mass of the lepton, neutrino, and first quark.

\subsection{Sampling from a Breit-Wigner mass distribution}

Sampling from a Breit-Wigner distribution is done by selecting a random point
between 0 and 1 from the cumulative distribution function of the BW
function. The cumulative distribution function, of cdf, for the Breit-Wigner
distribution is shown below.

\begin{equation}
\label{sample}
\int \sigma(m;m_{0};\Gamma) = F(m;m_{0},\Gamma) = \frac{1}{\pi} \tan^{-1} \left[
\frac{m-m_{0}}{\Gamma} \right] + \frac{1}{2}
\end{equation}

The value of F is taken as a random number between 0 and 1. After selecting a
value of F, the next step is to solve for m. As a function of F, defined as u for
the following, the mass is 


\begin{equation}
m = m_{0} + \Gamma \tan \left[ \pi(u - \frac{1}{2}) \right]
\end{equation}

\subsection{Sampling from a Breit-Wigner $S_{cm}$ distribution}

In the previous example, a distribution was sampled using a random number
uniformly distributed from 0 to 1. In, this example, a new random number is used
that is uniformaly samples from 0 to 1, but the maximum and minimum values of
the variable are taken into account in the Jacobian.

The distribution of the variable $S_{cm}$ is the following

\begin{equation}
\label{defines}
s = m_{0}^{2} + m_{0}\Gamma \tan \left[ m_{0} \Gamma r \right]
\end{equation}

where r is defined in terms of the random variable, u, that is uniformaly
distributed between 0 and 1.

\begin{equation}
\label{rtou}
r = (r_{max} - r_{min}) \times u + r_{min}
\end{equation}

where $r_{max}$ and $r_{min}$ are defined in terms of the variable $s_{cm}$.

\begin{eqnarray}
\label{definer}
r = \frac{1}{m_{0}\Gamma} \tan \left[ \frac{s - m_{0}^{2}}{m_{0}\Gamma} \right] \\
r_{min} = \frac{1}{m_{0}\Gamma} \tan \left[ \frac{s_{min} -
m_{0}^{2}}{m_{0}\Gamma} \right] \\
r_{max} = \frac{1}{m_{0}\Gamma} \tan \left[ \frac{s_{max} - m_{0}^{2}}{m_{0}\Gamma} \right]
\end{eqnarray}



\subsection{Jacobian for random sampling of a Breit-Wigner distribution around
the W mass squared, $s_{34}$}

The first case to consider is the Breit-Wigner sampling around the W mass
$S_{34}$ distribution and replace the integration variable, $p_{3}$ or the
lepton momentum.

\begin{equation}
\label{jacobian1}
|J(p_{3}, u)| = \left| \pderiv{p_{3}}{u} \right| 
\end{equation}

Because u is redined in terms of the variable r, we can rewrite ~\ref{jacobian1}
in terms of r instead of u.

\begin{equation}
\label{jacobian2}
|J(p_{3}, u)| = \left| \pderiv{p_{3}}{u} \right| = \left| \pderiv{p_{3}}{r}
\times \pderiv{r}{u} \right|
\end{equation}

And since the variable r is sampling the $S_{34}$ distribution it makes sense to
define the Jacobian in terms of this variable instead of $p_{3}$.

\begin{equation}
\label{jacobian3}
|J(p_{3}, u)| = \left| \pderiv{p_{3}}{r} \times \pderiv{r}{u} \right| = \left|
\pderiv{p_{3}}{s_{34}} \times \pderiv{s_{34}}{r} \times \pderiv{r}{u} \right| =
\left | \frac{\pderiv{s_{34}}{r} \times
\pderiv{r}{u}}{\pderiv{s_{34}}{p_{3}}} \right |
\end{equation}

Equation ~\ref{jacobian3} has three components: $\pderiv{s_{34}}{r}$,
$\pderiv{r}{u}$, and $\pderiv{s_{34}}{p_{3}}$. From equation ~\ref{rtou}, the
partial derivative of r with respect to u is 

\begin{equation}
\label{drdu}
\pderiv{r}{u} = r_{max} - r_{min} = \Delta r
\end{equation}

Next, the partial of $s_{34}$ with respect to r can be determined from equation
~\ref{defines}.


\begin{equation}
\label{ds34dr}
\pderiv{s_{34}}{r} = (m_{W} \Gamma_{W})^{2} \sec^{2} \left[m_{W} \Gamma_{W} r \right]
\end{equation}

Inserting the value of r(s) as defined in equation ~\ref{definer}, equation
~\ref{ds34dr} can be re-written as

\begin{equation}
\label{ds34dr_2}
\pderiv{s_{34}}{r} = (m_{W} \Gamma_{W})^{2} \sec^{2} \left[m_{W} \Gamma_{W} r \right] =
(m_{W} \Gamma_{W})^{2} \sec^{2} \left[\arctan \left[ \frac{s_{34} - m_{W}^{2}}{m_{W}
\Gamma_{W}}  \right] \right]
\end{equation}

Equation ~\ref{ds34dr_2} is solved by defining a right triangle where

\begin{eqnarray}
\nonumber
\tan(\theta) = \frac{s_{34}-m_{W}^{2}}{m_{W}\Gamma_{W}} \\
\cos(\theta) = \frac{1}{\sqrt{1+\left[ \frac{s_{34}-m_{W}^{2}}{m_{W}\Gamma_{W}}
\right]^{2}}}
\end{eqnarray}

Using these definitions, equation ~\ref{ds34dr_2} is finally defined as

\begin{equation}
\label{ds34dr_3}
\pderiv{s_{34}}{r} = (m_{W} \Gamma_{W})^{2} \sec^{2} \left[\arctan \left[ \frac{s_{34} - m_{W}^{2}}{m_{W}
\Gamma_{W}}  \right] \right] = (m_{W}\Gamma_{W})^{2} + (s_{34} - m_{W}^{2})^{2}
\end{equation}

Finally, we need the partial derivative of $s_{34}$ with repect to
$p_{3}$. First, we define $s_{34}$

\begin{equation}
\label{defines34}
s_{34} = m_{3}^{2} + m_{4}^{2} + 2E_{3}E_{4} - 2p_{3}^{x}p_{4}^{x} - 2p_{3}^{y}p_{4}^{y} - 2p_{3}^{z}p_{4}^{z}
\end{equation}

Since the neutino four-vector is defined in terms of all the other particles in
the event, we need to rewrite equation ~\ref{defines34} to expose all the
dependences on $p_{3}$. For the following, it is assumed that the lepton and
neutrino are massless meaning $E_{3} = p_{3}$. 

\begin{eqnarray}
\label{defines34_2}
\nonumber
s_{34} = 2p_{3}\sqrt{(-p_{3}^{x} -p_{5}^{x} -p_{6}^{x})^{2} + (-p_{3}^{y} -p_{5}^{y} -p_{6}^{y})^{2} + (p_{tot}^{z} -p_{3}^{z} -p_{5}^{z} -p_{6}^{z})^{2}} \\
- 2p_{3}^{x}(-p_{3}^{x} -p_{5}^{x} -p_{6}^{x}) -
2p_{3}^{y}(-p_{3}^{y} -p_{5}^{y} -p_{6}^{y}) - 2p_{3}^{z}(p_{tot}^{z} -p_{3}^{z} -p_{5}^{z} -p_{6}^{z})
\end{eqnarray}

After combining like terms, we can evaluate the partial derivative of $s_{34}$
with respect to $p_{3}$ that yields the relatively
simple formula

\begin{equation}
\label{ds34dp3}
\pderiv{s_{34}}{p_{3}} = 2(p_{3} + p_{4})(1 - \hat{p_{3}} \cdot \hat{p_{4}})
\end{equation}

Finally, we can rewrite the Jacobian defined in ~\ref{jacobian3} as

\begin{equation}
\label{jacobian4}
|J(p_{3}, u)| = \left | \frac{\pderiv{s_{34}}{r} \times
\pderiv{r}{u}}{\pderiv{s_{34}}{p_{3}}} \right | = \frac{\Delta R \times \left[
(m_{W}\Gamma_{W})^{2} + (s_{34} - m_{W}^{2})^{2} \right]}{2(p_{3} + p_{4})(1 - \hat{p_{3}} \cdot \hat{p_{4}})}
\end{equation}

In some cases, it is also common to replace the first quark momentum integration
with the Breit-Wigner sampling variable. In that case, we need to evaluate

\begin{equation}
\label{jacobian_5}
|J(p_{5}, u)| = \left | \frac{\pderiv{s_{34}}{r} \times
\pderiv{r}{u}}{\pderiv{s_{34}}{p_{5}}} \right |
\end{equation}

For this substitution, we only need to evaluate the partial derivative of
$s_{34}$ with respect to $p_{5}$. Assuming a massless quark, the result is

\begin{equation}
\label{ds34dp5}
\pderiv{s_{34}}{p_{5}} = 2p_{3}(\hat{p_{3}} \cdot \hat{p_{5}} - \hat{p_{4}} \cdot \hat{p_{5}})
\end{equation}

Combining equation ~\ref{ds34dp5} with ~\ref{jacobian_5} yields

\begin{equation}
\label{jacobian_6}
|J(p_{5}, u)| = \left | \frac{\pderiv{s_{34}}{r} \times
\pderiv{r}{u}}{\pderiv{s_{34}}{p_{5}}} \right | = \frac{\Delta R \times \left[
(m_{W}\Gamma_{W})^{2} + (s_{34} - m_{W}^{2})^{2} \right]}{2p_{3}(\hat{p_{3}} \cdot \hat{p_{5}} - \hat{p_{4}} \cdot \hat{p_{5}})}
\end{equation}



%
%-----------------------------------------------------------------------
%



\subsection{Jacobian for random sampling of a Breit-Wigner distribution around
the top mass squared, $s_{345}$}

The next case to consider is the Breit-Wigner sampling around the top mass squared
$S_{345}$ distribution and replace the integration variable, $p_{3}$ or the
lepton momentum. As before, we need need to calculate the following

\begin{equation}
\label{jacobian_1}
|J(p_{3}, u)| = \left| \pderiv{p_{3}}{r} \times \pderiv{r}{u} \right| = \left|
\pderiv{p_{3}}{s_{345}} \times \pderiv{s_{345}}{r} \times \pderiv{r}{u} \right| =
\left | \frac{\pderiv{s_{345}}{r} \times
\pderiv{r}{u}}{\pderiv{s_{345}}{p_{3}}} \right |
\end{equation}

Equation ~\ref{jacobian_1} has three components: $\pderiv{s_{345}}{r}$,
$\pderiv{r}{u}$, and $\pderiv{s_{345}}{p_{3}}$. We know $\pderiv{r}{u}$ from
equation ~\ref{drdu} where $r_{min}$ and $r_{max}$ are defined by the $s_{345}$
system instead of the $s_{34}$ system. We also know $\pderiv{s_{345}}{r}$ from ~\ref{ds34dr_3}
where we replace $s_{34}$ with $s_{345}$.

\begin{equation}
\label{ds345dr}
\pderiv{s_{345}}{r} = (m_{t}\Gamma_{t})^{2} + (s_{345} - m_{t}^{2})^{2}
\end{equation}

We do need the partial derivative of $s_{345}$ with repect to
$p_{3}$. First, we define $s_{345}$

\begin{eqnarray}
\label{defines345}
\nonumber
s_{345} = m_{3}^{2} + m_{4}^{2} + m_{5}^{2} + 2E_{3}E_{4} + 2E_{3}E_{5} +
2E_{4}E_{5} - \\
2p_{3}^{x}p_{4}^{x} - 2p_{3}^{x}p_{5}^{x} - 2p_{4}^{x}p_{5}^{x} -
2p_{3}^{y}p_{4}^{y} - 2p_{3}^{y}p_{5}^{y} - 2p_{4}^{y}p_{5}^{y} -
2p_{3}^{z}p_{4}^{z} - 2p_{3}^{z}p_{5}^{z} - 2p_{4}^{z}p_{5}^{z}
\end{eqnarray}

As before, the neutino four-vector is defined in terms of all the other particles in
the event so we need to rewrite equation ~\ref{defines345} to expose all the
dependences on $p_{3}$. For the following, it is assumed that the lepton,
neutrino, and quark are massless meaning $E_{3} = p_{3}$.

\begin{eqnarray}
\label{defines345_2}
\nonumber
s_{345} = 2p_{3}\sqrt{(-p_{3}^{x} -p_{5}^{x} -p_{6}^{x})^{2} + (-p_{3}^{y}
-p_{5}^{y} -p_{6}^{y})^{2} + (p_{tot}^{z} -p_{3}^{z} -p_{5}^{z} -p_{6}^{z})^{2}}
+ \\
\nonumber
2p_{3}p_{5} + 2\sqrt{(-p_{3}^{x} -p_{5}^{x} -p_{6}^{x})^{2} + (-p_{3}^{y} -p_{5}^{y}
-p_{6}^{y})^{2} + (p_{tot}^{z} -p_{3}^{z} -p_{5}^{z} -p_{6}^{z})^{2}}p_{5} - \\
\nonumber
2p_{3}^{x}(-p_{3}^{x} -p_{5}^{x} -p_{6}^{x}) - 2p_{3}^{x}p_{5}^{x} -
2(-p_{3}^{x} -p_{5}^{x} -p_{6}^{x})p_{5}^{x} - \\
\nonumber
2p_{3}^{y}(-p_{3}^{y} -p_{5}^{y} -p_{6}^{y}) - 2p_{3}^{y}p_{5}^{y} -
2(-p_{3}^{y} -p_{5}^{y} -p_{6}^{y})p_{5}^{y} - \\
2p_{3}^{z}(-p_{3}^{z} -p_{5}^{z} -p_{6}^{z}) - 2p_{3}^{z}p_{5}^{z} - 2(-p_{3}^{z} -p_{5}^{z} -p_{6}^{z})p_{5}^{z}
\end{eqnarray}

After combining like terms, we can evaluate the partial derivative of $s_{345}$
with respect to $p_{3}$ as

\begin{equation}
\label{ds345dp3}
\pderiv{s_{345}}{p_{3}} = 2(p_{3} + p_{4} + p_{5})(1 - \hat{p_{3}} \cdot \hat{p_{4}})
\end{equation}

Finally, we can rewrite the Jacobian defined in ~\ref{jacobian_1} as

\begin{equation}
\label{jacobian_2}
|J(p_{3}, u)| = \left | \frac{\pderiv{s_{345}}{r} \times
\pderiv{r}{u}}{\pderiv{s_{345}}{p_{3}}} \right | = \frac{\Delta R \times \left[
(m_{t}\Gamma_{t})^{2} + (s_{345} - m_{t}^{2})^{2} \right]}{2(p_{3} + p_{4} + p_{5})(1 - \hat{p_{3}} \cdot \hat{p_{4}})}
\end{equation}


Instead of replacing the lepton momentum integration variable, it is also common
to replace the first quark momentum integration variable, $p_{5}$. In that case,
we need to evaluate the following Jacobian.

\begin{equation}
\label{jacobian_3}
|J(p_{5}, u)| = \left| \pderiv{p_{5}}{r} \times \pderiv{r}{u} \right| = \left|
\pderiv{p_{5}}{s_{345}} \times \pderiv{s_{345}}{r} \times \pderiv{r}{u} \right| =
\left | \frac{\pderiv{s_{345}}{r} \times
\pderiv{r}{u}}{\pderiv{s_{345}}{p_{5}}} \right |
\end{equation}

The only difference is that partial derivative of $s_{345}$ with respect to
$p_{5}$ instead of $p_{3}$. However, since $s_{345}$ is invariant under a change
of $p_{3}$ and $p_{5}$ the partial derivatives must be equal. Thus, 

\begin{equation}
\label{jacobian_4}
|J(p_{5}, u)| = \left | \frac{\pderiv{s_{345}}{r} \times
\pderiv{r}{u}}{\pderiv{s_{345}}{p_{5}}} \right | = \frac{\Delta R \times \left[
(m_{t}\Gamma_{t})^{2} + (s_{345} - m_{t}^{2})^{2} \right]}{2(p_{3} + p_{4} + p_{5})(1 - \hat{p_{4}} \cdot \hat{p_{5}})}
\end{equation}





%
%-----------------------------------------------------------------------
%


\subsection{Jacobian for random sampling of two Breit-Wigner distributions around
the top mass squared, $s_{345}$ and W mass squared, $s_{34}$}


The next situation is to sample from a Breit-Wigner around the
top mass squared and the W mass squared, or $s_{345}$ and $s_{34}$. It is common
to replace the lepton momentum and first quark momentum integration variables
with the two new variables. Since we are
replacing two variables, we need to evaluate the following Jacobian

\begin{equation}
|J(p_{3}, p_{5} ; u_{1}, u_{2})| = \left| \begin{array}{cc}
\pderiv{p_{3}}{u_{1}}	& \pderiv{p_{3}}{u_{2}} \\
\pderiv{p_{5}}{u_{1}}	& \pderiv{p_{5}}{u_{2}} \\
\end{array} \right|
\end{equation}

where $u_{1}$ and $u_{2}$ are the sampling variables around the top mass squared
and W mass squared, respectively.

We have already computed the partial derivatives for each of these cases
in the previous two sections, thus the result is

\begin{eqnarray}
\nonumber
|J(p_{3}, p_{5} ; u_{1}, u_{2})| = \left| \begin{array}{cc}
\pderiv{p_{3}}{u_{1}}	& \pderiv{p_{3}}{u_{2}} \\
\pderiv{p_{5}}{u_{1}}	& \pderiv{p_{5}}{u_{2}} \\
\end{array} \right| = 
\nonumber
\left| \begin{array}{cc}
\frac{\pderiv{s_{345}}{r} \times \pderiv{r}{u}}{\pderiv{s_{345}}{p_{3}}}	& \frac{\pderiv{s_{345}}{r} \times
\pderiv{r}{u}}{\pderiv{s_{345}}{p_{5}}} \\
\frac{\pderiv{s_{34}}{r} \times \pderiv{r}{u}}{\pderiv{s_{34}}{p_{3}}}	& \frac{\pderiv{s_{34}}{r} \times
\pderiv{r}{u}}{\pderiv{s_{34}}{p_{5}}}
\end{array} \right| = \\
\left| \begin{array}{cc}
\frac{\Delta R_{345} \times \left[
(m_{t}\Gamma_{t})^{2} + (s_{345} - m_{t}^{2})^{2} \right]}{2(p_{3} + p_{4} + p_{5})(1 - \hat{p_{3}} \cdot \hat{p_{4}})}	& \frac{\Delta R_{34} \times \left[
(m_{W}\Gamma_{W})^{2} + (s_{34} - m_{W}^{2})^{2} \right]}{2(p_{3} + p_{4})(1 -
\hat{p_{3}} \cdot \hat{p_{4}})} \\
\frac{\Delta R_{345} \times \left[
(m_{t}\Gamma_{t})^{2} + (s_{345} - m_{t}^{2})^{2} \right]}{2(p_{3} + p_{4} + p_{5})(1 - \hat{p_{4}} \cdot \hat{p_{5}})}	& \frac{\Delta R_{34} \times \left[
(m_{W}\Gamma_{W})^{2} + (s_{34} - m_{0}^{2})^{2} \right]}{2p_{3}(\hat{p_{3}} \cdot \hat{p_{5}} - \hat{p_{4}} \cdot \hat{p_{5}})}
\end{array}
\right|
\end{eqnarray}


\subsection{Sampling from a Polynomial $S_{cm}$ distribution}

*** This is where I am taking a function from Aurelio and I can't seem to derive
it on my own ***

The distribution of the variable $S_{cm}$ according to a polynomial power
distribution is

\begin{equation}
\label{defines_power}
s = m_{0}^{2} + \left[ (1-\alpha)r \right]^{\frac{-1}{\alpha - 1}}
\end{equation}

where r is defined in terms of the random variable, u, that is uniformaly
distributed between 0 and 1.

\begin{equation}
\label{rtou_power}
r = (r_{max} - r_{min}) \times u + r_{min}
\end{equation}

where $r_{max}$ and $r_{min}$ are defined in terms of the variable $s_{cm}$.

\begin{eqnarray}
\label{definer_power}
r = \frac{1}{1-\alpha} \times \left[ s - m_{0}^{2} \right]^{1-\alpha} \\
r_{min} = \frac{1}{1-\alpha} \times \left[ s_{min} - m_{0}^{2}
\right]^{1-\alpha} \\
r_{max} = \frac{1}{1-\alpha} \times \left[ s_{max} - m_{0}^{2}
\right]^{1-\alpha}
\end{eqnarray}

where alpha can not equal 1.


\subsection{Jacobian for random sampling of a polynomial distribution
starting at $m_{pole}$}

The first case to consider is sampling around a falling polynomial distribution
for the mass squared of two quarks in the event, $s_{56}$. We need to define the
Jacobian with respect $p_{5}$ or $p_{6}$. Since $s_{56}$ is invariant under an
interchange of particle 5 and 6, the Jacobian will be the same for each momentum
integration. The following assume $p_{5}$ will be replaced with the variable,
$u$, which is sampled from a polynomial distribution.

\begin{equation}
\label{jacobian_7}
|J(p_{5}, u)| = \left| \pderiv{p_{5}}{u} \right| 
\end{equation}

Because u is redined in terms of the variable r, we can rewrite ~\ref{jacobian_7}
in terms of r instead of u.

\begin{equation}
\label{jacobian_8}
|J(p_{5}, u)| = \left| \pderiv{p_{5}}{u} \right| = \left| \pderiv{p_{5}}{r}
\times \pderiv{r}{u} \right|
\end{equation}

And since the variable r is sampling the $S_{56}$ distribution it makes sense to
define the Jacobian in terms of this variable instead of $p_{5}$.

\begin{equation}
\label{jacobian_9}
|J(p_{5}, u)| = \left| \pderiv{p_{5}}{r} \times \pderiv{r}{u} \right| = \left|
\pderiv{p_{5}}{s_{56}} \times \pderiv{s_{56}}{r} \times \pderiv{r}{u} \right| =
\left | \frac{\pderiv{s_{56}}{r} \times
\pderiv{r}{u}}{\pderiv{s_{56}}{p_{5}}} \right |
\end{equation}

Equation ~\ref{jacobian_9} has three components: $\pderiv{s_{56}}{r}$,
$\pderiv{r}{u}$, and $\pderiv{s_{56}}{p_{5}}$. From equation ~\ref{rtou_power}, the
partial derivative of r with respect to u is 

\begin{equation}
\label{drdu_power}
\pderiv{r}{u} = r_{max} - r_{min} = \Delta R_{56}
\end{equation}

Next, the partial of $s_{56}$ with respect to r can be determined from equation ~\ref{defines_power}.

\begin{equation}
\label{ds56dr_power}
\pderiv{s_{56}}{r} = \left[ r(1-\alpha) \right]^{\frac{\alpha}{1-\alpha}}
\end{equation}

Inserting the value of r(s) as defined in equation ~\ref{definer_power}, equation ~\ref{ds56dr_power} can be re-written as

\begin{equation}
\label{ds56dr_power2}
\pderiv{s_{56}}{r} = \left[ s_{56} - m_{0}^{2} \right]^{\alpha}
\end{equation}

Finally, we need the partial derivative of $s_{56}$ with repect to $p_{5}$. First, we define $s_{56}$

\begin{equation}
\label{defines56_power}
s_{56} = m_{5}^{2} + m_{6}^{2} + 2E_{5}E_{6} - 2p_{5}^{x}p_{6}^{x} - 2p_{5}^{y}p_{6}^{y} - 2p_{5}^{z}p_{6}^{z}
\end{equation}

we can evaluate the partial derivative of $s_{56}$ with respect to $p_{5}$.

\begin{equation}
\label{ds56dp5_power}
\pderiv{s_{56}}{p_{5}} = 2p_{6}(1 - \hat{p_{5}} \cdot \hat{p_{6}})
\end{equation}

Finally, we can rewrite the Jacobian defined in ~\ref{jacobian_9} as

\begin{equation}
\label{jacobian_10}
|J(p_{5}, u)| = \left | \frac{\pderiv{s_{56}}{r} \times
\pderiv{r}{u}}{\pderiv{s_{56}}{p_{5}}} \right | = \frac{\Delta R_{56} \times \left[ s_{56}-m_{0}^{2} \right]^{\alpha}}{2p_{6}(1 - \hat{p_{5}} \cdot \hat{p_{6}})}
\end{equation}

and 

\begin{equation}
\label{jacobian_11}
|J(p_{6}, u)| = \left | \frac{\pderiv{s_{56}}{r} \times
\pderiv{r}{u}}{\pderiv{s_{56}}{p_{6}}} \right | = \frac{\Delta R_{56} \times \left[ s_{56}-m_{0}^{2} \right]^{\alpha}}{2p_{5}(1 - \hat{p_{5}} \cdot \hat{p_{6}})}
\end{equation}


%---------------------------------------------------------------------
\section{Probability Calculation}
\label{full}

\subsection{Differential Cross Section at the Parton Level}

The matrix element analysis technique reconstructs each event to the
final state four-vectors to evaluate the signal and background leading
order matrix element. The following sections derive the signal and
background probabilities starting from the final state at the parton
level and then relating these objects to the physical quantities
measured in the detector. The following also assumes a lepton,
neutrino, and two quarks in the final state.

The probability density for a process to occur at a hadron-hadron
collider is given as an integral of the hard-scatter differential
cross section over all possible ways of producing the process from the
quarks and gluons inside the hadron. This probability density, shown
below, is a convolution of the hard scatter differential cross section
with a parton distribution function for each of the two partons from
the hadrons with an integral over all possible momentum fractions
$x_{i}, x_{j}$ from each initial parton.
\begin{equation}
{\cal P}(\vec{y}) = \frac{1}{\sigma} \sum_{i,j}
\int f_{i}(q_{1}, Q^{2})dq_{1}
\times f_{j}(q_{2}, Q^{2})dq_{2}
\times d\sigma_{hs,ij}(\vec{y})
\end{equation}
\noindent where the normalization constant $\sigma$ is defined as
integral of the differential cross section over the initial- and
final-state phase spac:
\begin{equation}
\label{ds}
\sigma = \int  \sum_{i,j}
\int f_{i}(q_{1}, Q^{2})dq_{1}
\times f_{j}(q_{2}, Q^{2})dq_{2}
\times \pderiv{\sigma_{hs,ij}(\vec{y})}{\vec{y}} d\vec{y}
\end{equation}
\noindent and finally, the hard-scatter differential cross section is
defined as the product of the final state phase space factor, the
square of the matrix element amplitude and an overall flux factor:
\begin{equation}
d\sigma_{hs} = \frac{(2\pi)^4}{4}
\frac{{|\cal M|}^{2}}
{\sqrt{(q_{1}q_{2})^2 - m_{1}^2 m_{2}^2}}
\frac{d^{3}p_{1}}{(2\pi)^3 2E_{1}}
\frac{d^{3}p_{2}}{(2\pi)^3 2E_{2}}
\frac{d^{3}p_{\ell}}{(2\pi)^3 2E_{\ell}}
\frac{d^{3}p_{\nu}}{(2\pi)^3 2E_{\nu}}
\delta^{4}(q_{1}q_{2};p_{1},p_{2},p_{\ell},p_{\nu})
\end{equation}


\subsection{Evaluating the Hard Scatter Differential Cross Section}

The following section evaluates the differential cross section shown
in Eq.~\ref{ds} given a set of inital and final state four-vectors.

The first assumption made is that all collisions occur along the beam
axis with no net transverse momentum. This means the initial state
four vectors can be written as
\begin{equation}
q_{1} = ( E_{beam} x_{1}, 0, 0, E_{beam} x_{1} )
\end{equation}
\begin{equation}
q_{2} = ( E_{beam} x_{2}, 0, 0, -E_{beam} x_{2} )
\end{equation}
\noindent The next assumption is that all particle masses are known
and are negligible compared to their energies and thus can be ignored
for this calculation. The flux factor (shown below) in the hard
scatter cross section can now be written in terms in the two momentum
fractions of the incoming partons:
\begin{equation}
\frac{1}{\sqrt{(q_{1}q_{2})^2 - m_{1}^2m_{2}^2}} \rightarrow
\frac{1}{\sqrt{(q_{1}q_{2})^2}} \rightarrow
\frac{1}{2E_{beam}x_{1}x_{2}}
\end{equation}
\noindent For the remainder of the note, the following notation will
be used to distinquish quarks, leptons, and neutrinos: $p_{\ell}$ is
the momemtum of the lepton, $p_{1,2}$ is the momentum of the first and
second final state partons, and $p_{\nu}$ is the neutrino
momentum. Because the phase space is written in terms of rectangular
coordinates, the next step towards the final differential cross
section equation is to redefine the phase space factors in terms of
spherical coordiniates. This is done for all final state particles
except the neutrino for reasons that will be clear later in the
document.
\begin{eqnarray}
d\Phi_{4} = 
\frac{d^{3}p_{1}}{(2\pi)^3 2E_{1}}
\frac{d^{3}p_{2}}{(2\pi)^3 2E_{2}}
\frac{d^{3}p_{\ell}}{(2\pi)^3 2E_{\ell}}
\frac{d^{3}p_{\nu}}{(2\pi)^3 2E_{\nu}} \delta^{4} 
\rightarrow \\
\frac{1}{16(2\pi)^{12}}
\frac{|p_{1}|^{2}d|p_{1}|d\Omega_{1}}{E_{1}}
\frac{|p_{2}|^{2}d|p_{2}|d\Omega_{2}}{E_{2}}
\frac{|p_{\ell}|^{2}d|p_{\ell}|d\Omega_{\ell}}{E_{\ell}}
\frac{d^{x}_{\nu} d^{y}_{\nu} d^{z}_{\nu}}{E_{\nu}}\delta^{4} 
%\frac{d^{3}p_{1}}{(2\pi)^3 2E_{1}} \frac{d^{3}p_{2}}{(2\pi)^32E_{2}} \frac{d^{3}p_{\ell}}{(2\pi)^3 2E_{\ell}}
%\frac{d^{3}p_{\nu}}{(2\pi)^3 2E_{\nu}} \delta^{4} \righarrow 
%\frac{1}{16(2\pi)^{12}}
\end{eqnarray}
\noindent To summarize, the full hard scatter differential cross
section is now
\begin{equation}
\label{dhs}
d\sigma_{hs} = \frac{1}{128(2\pi)^{8}E_{beam}}
\frac{{|\cal M|}^{2}}{2x_{1}x_{2}}
\frac{|p_{1}|^{2}d|p_{1}|d\Omega_{1}}{E_{1}}
\frac{|p_{2}|^{2}d|p_{2}|d\Omega_{2}}{E_{2}}
\frac{|p_{\ell}|^{2}d|p_{\ell}|d\Omega_{\ell}}{E_{\ell}}
\frac{d^{x}_{\nu} d^{y}_{\nu} d^{z}_{\nu}}{E_{\nu}} \delta^{4} 
\end{equation} 


\subsection{Evaluating the Hadron-Hadron Differential Cross Section}

The next step to writing the full hadron-hadron differential cross
section is to rewrite Eq.~\ref{dhs} such that any integration over the
phase space will remove the four-dimensional delta function required
for energy and momentum conservation. The delta function is currently
written such that it will vanish only over integrations of total
$p_{x}, p_{y}, p_{z}$, and $E$. Because there was an original
assumption of no net transverse momentum in the collision, the total
$p_{x}$ and $p_{y}$ can be solved for the neutrino transverse
momentum.
\begin{eqnarray}
\sum_{i}P^{x}_{i} =
p^{x}_{1} + p^{x}_{2} + p^{x}_{\ell} + p^{x}_{\nu} = 0
\rightarrow p^{x}_{\nu} =
- p^{x}_{1} - p^{x}_{2} - p^{x}_{\ell} \\
\sum_{i}P^{y}_{i} =
p^{y}_{1} + p^{y}_{2} + p^{y}_{\ell} + p^{y}_{\nu} = 0
\rightarrow p^{y}_{\nu} =
- p^{y}_{1} - p^{y}_{2} - p^{y}_{\ell}
\end{eqnarray}
\noindent The total $p_{z}$ and requirement can be rewritten in terms
of the intial parton's momemtum fraction and the other final state
partons' $z$ momenta and thus solve for the neutrino $p_{z}$:
\begin{eqnarray}
\nonumber
\sum_{i}P^{z}_{i} =
p^{z}_{1} + p^{z}_{2} + p^{z}_{\ell} + p^{z}_{\nu} -
E_{beam}x_{1} + E_{beam}x_{2} = 0 \rightarrow \\
p^{z}_{\nu} =
- E_{beam}(x_{1} - x_{2}) - p^{z}_{1} - p^{z}_{2} - p^{z}_{\ell}
\end{eqnarray}
\noindent Finally, the total energy delta function implies the
following:
\begin{equation}
E_{beam}x_{1} + E_{beam}x_{2} = E_{1} + E_{2} + E_{\ell} + E_{\nu}
\end{equation}

At this point, is it useful to rewrite the full differential cross
section at the parton level:
\begin{eqnarray}
\nonumber
d\sigma(\vec{y}) = \sum_{i,j} \int f_{i}(x_{1}, Q^{2})dx_{1}
\times f_{j}(x_{2}, Q^{2})dx_{2}
\times \frac{1}{128(2\pi)^{8}E_{beam}}
\frac{{|\cal M|}^{2}}{2x_{1}x_{2}} \times \\
\nonumber
\frac{|p_{1}|^{2}d|p_{1}|d\Omega_{1}}{E_{1}}
\frac{|p_{2}|^{2}d|p_{2}|d\Omega_{2}}{E_{2}}
\frac{|p_{\ell}|^{2}d|p_{\ell}|d\Omega_{\ell}}{E_{\ell}}
\frac{d^{x}_{\nu} d^{y}_{\nu} d^{z}_{\nu}}{E_{\nu}} \times \\
\nonumber
\delta(p^{x}_{\nu} + p^{x}_{1} + p^{x}_{2} + p^{x}_{\ell}) \times \\
\nonumber
\delta(p^{y}_{\nu} + p^{y}_{1} + p^{y}_{2} + p^{y}_{\ell}) \times \\
\nonumber
\delta(p^{z}_{\nu} + E_{beam}(x_{1} - x_{2}) + p^{z}_{1}
+ p^{z}_{2} + p^{z}_{\ell}) \times \\
\delta(E_{beam}x_{1} + E_{beam}x_{2} - E_{1} - E_{2}
- E_{\ell} - E_{\nu})
\end{eqnarray}

The next step is to rewrite the integrational variables, $x_{1}$ and
$x_{2}$, in terms of the total energy and total $p_{z}$:
\begin{eqnarray}
\label{x1}
x_{1} = \frac{E_{tot} + p^{z}_{tot}}{2E_{beam}} \\
\label{x2}
x_{2} = \frac{E_{tot} - p^{z}_{tot}}{2E_{beam}}
\end{eqnarray}
\noindent Now, the integration over $x_{1}$ and $x_{2}$ can be
rewritten in terms of $E_{tot}$ and $p_{z}$:
\begin{eqnarray}
dx_{1}dx_{2} =
\frac{1}{J(x_{1},x_{2};E_{tot},p^{z}_{tot})}dE_{tot}dp^{z}_{tot} \\
J(x_{1},x_{2};E_{tot},p^{z}_{tot}) = 2E^{2}_{beam}
\end{eqnarray}
\noindent At this point the integration over the total energy and
$p_{z}$ will constrain the two incoming partons' momentum fractions
through Eq.~\ref{x1} and \ref{x2}.

The full differential cross section at the parton level can now be
written as
\begin{eqnarray}
\nonumber
d\sigma(\vec{y}) = \sum_{i,j} \int f_{i}(x_{1}, Q^{2})
\times 
f_{j}(x_{2}, Q^{2}) \times \frac{1}{128(2\pi)^{8}E_{beam}}
\frac{{|\cal M|}^{2}}{2x_{1}x_{2}} \times \\
\nonumber
\frac{|p_{1}|^{2}d|p_{1}|d\Omega_{1}}{E_{1}}
\frac{|p_{2}|^{2}d|p_{2}|d\Omega_{2}}{E_{2}}
\frac{|p_{\ell}|^{2}d|p_{\ell}|d\Omega_{\ell}}{E_{\ell}}
\frac{d^{x}_{\nu} d^{y}_{\nu} d^{z}_{\nu}}{E_{\nu}} \times \\
\int \frac{1}{2E^{2}_{beam}} dp^{z}_{tot}
\end{eqnarray}
\noindent where the implicit integration over the four dimensional
delta function yields the following formulas for the neutrino four
vector and the incoming partons' momentum fraction in terms of the
remaining differential variables.
\begin{eqnarray}
p^{x}_{\nu} = - p^{x}_{1} - p^{x}_{2} - p^{x}_{\ell} \\
p^{y}_{\nu} = - p^{y}_{1} - p^{y}_{2} - p^{y}_{\ell} \\
p^{z}_{\nu} = - p^{z}_{tot} - p^{z}_{1} - p^{z}_{2} - p^{z}_{\ell} \\
x_{1} = \frac{E_{1} + E_{2} + E_{\ell} + E_{\nu} + p^{z}_{tot}}
{2E_{beam}} \\
x_{2} = \frac{E_{1} + E_{2} + E_{\ell} + E_{\nu} - p^{z}_{tot}}
{2E_{beam}}
\end{eqnarray}


\subsection{Relating Reconstructed Objects to Partons}

The previous sections have calculated the differential cross section
for a hadron-hadron collision producing a lepton, neutrino, and two
partons in the final state. These particles are not exactly what is
measured in the detector and thus it is necessary to relate
quantities. To do this, the differential cross section is convoluted
with a function, $W(\vec{x}, \vec{y})$, which is the probability of
producing a final state, $\vec{y}$, and observed state, $\vec{x}$, in
the detector. The resulting differential cross section is then
integrated over the final state phase space, $d\vec{y}$:
\begin{equation}
\pderiv{\sigma^{'}(\vec{x})}{\vec{x}} =
\int \pderiv{\sigma(\vec{y})}{\vec{y}} W(\vec{x}, \vec{y}) d\vec{y}
\end{equation}
\noindent where the function $W(\vec{x}, \vec{y})$ is assumed to be
factorizable for each measured object:
\begin{equation}
W(\vec{x}, \vec{y}) = \prod_{i} W_{i}(\vec{x_{i}}, \vec{y_{i}})
\end{equation}


\subsubsection{Jets}

The transfer function for jets measured in the calorimter is assumed
to only be a function of the relative energy difference between the
two objects and all angles are assumed to be well measured:
\begin{equation}
W_{jet}(\vec{x}_{jet}, \vec{y}_{parton}) = W(E_{jet} -
E_{parton}) \times \delta(\Omega_{jet} - \Omega_{parton})
\end{equation}
\noindent where $W(E_{jet} - E_{parton})$ is parametrized using the
following functional form:
\begin{equation}
W(E_{jet} - E_{parton}) = \frac{e^{-\frac{(E_{jet}
- E_{parton} - p_{1})^{2}}{2p^{2}_{2}}} + p_{3}e^{-\frac{(E_{jet}
- E_{parton} - p_{4})^{2}}{2p^{5}_{2}}}}{2\pi(p_{2} + p_{3}p_{5})}
\end{equation}
\noindent where $p_{i} = \alpha_{i} + \beta_{i} \times E_{parton}$.
The five $\alpha$ and five $\beta$ parameters are determined by
minimizing a likelihood formed by measuring the parton energy in Monte
Carlo and the matched jet energy also in Monte Carlo. The parameters
used for this analysis were determined in several regions of the
calorimeter to account for the resolution differences in the detector.


\subsubsection{Electrons}

The transfer function for electrons~\cite{ElectronTF} is assumed to be
solely a function of the reconstructed energy of the electron,
$E_{reco}$, the parton energy of the electron, $E_{parton}$, and
$\theta$, the production angle with respect to the beam axis:
\begin{equation}
W_{electron}(\vec{x}_{reco}, \vec{y}_{parton}) = W(E_{reco},
E_{parton}, \theta) \times \delta(\Omega_{reco} - \Omega_{parton})
\end{equation}
\noindent where $W(E_{reco}, E_{parton}, \theta)$ is parametrized
using the following functional form:
\begin{eqnarray}
W(E_{reco}, E_{parton}, \theta) & = &
\frac{1}{2\pi\sigma}\mathrm{exp}
[-\frac{(E_{reco} - E_{center})^{2}}{2\sigma^{2}}]\\
E_{center} & = &
1.0002 E_{parton} + 0.324\,{GeV}/c^2 \\
\sigma & = &
0.028 E_{center} \oplus \textrm{Sampling}(E_{center},
\eta) E_{center} \oplus 0.4 \\
\textrm{Sampling}(E, \theta) & = &
\left[\frac{0.164}{\sqrt{E}} + \frac{0.122}{E}\right]
\textrm{exp}\left[\frac{\textrm{p1}(E)}
{\textrm{sin}\theta}-\textrm{p1}(E)\right] \\
\textrm{p1}(E)& = & 1.35193 - \frac{2.09564}{E} - \frac{6.98578}{E^2}.
\end{eqnarray}


\subsubsection{Muons}

The transfer function for muons~\cite{MuonTF} is assumed to be a
function of
\begin{equation}
\Delta\left( \frac{q}{p_t}\right) =
\left( \frac{q}{p_t}\right)_{reco} -
\left( \frac{q}{p_t}\right)_{parton}
\end{equation}
\noindent and of $\eta_\mathrm{CFT}$,
\begin{equation}
W_{muon}(\vec{x}_{reco}, \vec{y}_{parton}) =
W\left(\Delta\left( \frac{q}{p_t}\right),
\eta_\mathrm{CFT}\right) \times 
\delta(\Omega_{reco} - \Omega_{parton})
\end{equation}
\noindent where $W\left(\Delta\left( \frac{q}{p_t}\right),
\eta_\mathrm{CFT}\right)$ is parametrized using a single Gaussian:
\begin{equation}
W\left(\Delta\left( \frac{q}{p_t}\right), \eta_\mathrm{CFT}\right) = 
\frac{1}{2\pi\sigma}\mathrm{exp}
\left\{-\frac{\left[\Delta
\left( \frac{q}{p_t} \right)\right]^2}
{2\sigma^2}\right\}
\end{equation}
\begin{equation}
\sigma  =  \left\{ 
\begin{array} {c@{\quad:\quad}l} \sigma_o &
|\eta_\mathrm{CFT}| \le \eta_o \\
\sqrt{\sigma^2_o + [c(|\eta_\mathrm{CFT}| - \eta_o)]^2} &
|\eta_\mathrm{CFT}| > \eta_o 
\end{array} \right.
\end{equation}

\noindent There are three fitted parameters in the above equations:
$\sigma_o$, $c$, and $\eta_o$, each of which is actually fitted by two
sub-parameters:
\begin{equation}
par = par(0) + par(1) \cdot 1/p_t.
\end{equation}
\noindent Furthermore, these parameters are derived for four classes
of events: those that were from before or after the 2004 shutdown,
when the magnetic field strength changed, and in each run range, those
that have an SMT hit and those that do not.

As a simplification, we assume $q_{reco} = q_{parton}$, that is, we
do not consider charge misidentification


\subsection{Full Differential Cross Section and Normalization}

The full differential cross section at the detector object level can
now be written as
\begin{eqnarray}
\label{fullds}
\nonumber
\pderiv{\sigma^{'}(\vec{x})}{\vec{x}} =
\int dp^{z}_{tot}dq_{1}dq_{2}dp_{\ell} \sum_{i,j}
f_{i}(q_{1}, Q^{2}) \times f_{j}(q_{2}, Q^{2}) \\
\times \frac{1}{256(2\pi)^{8}E^{3}_{beam}}
\frac{{|\cal M|}^{2}}{2x_{1}x_{2}} \times
\frac{p_{1}^{2}}{E_{1}} \frac{p_{2}^{2}}{E_{2}} 
\frac{p_{\ell}^{2}}{E_{\ell}} \frac{1}{E_{\nu}}
\times W_{Lepton}W_{Jet1}W_{Jet2}
\end{eqnarray}
\noindent The final step to evaluating the probability density is to
properly normalize the differential cross section in
Eq.~\ref{fullds}. This is done by integration of the differential
cross section over all possible states in the detector. Since the
event selection cuts will change the number events due to acceptance
losses, this must be accounted for in the overall normalization (cross
section) calculation. The total cross section is then written as
\begin{eqnarray}
\label{norm}
\nonumber
\sigma = \int \pderiv{\sigma^{'}(\vec{x})}{\vec{x}} d\vec{x} =
\int d\vec{x}dp^{z}_{tot}dq_{1}dq_{2}dp_{\ell} \sum_{i,j}
f_{i}(q_{1}, Q^{2}) \times f_{j}(q_{2}, Q^{2}) \\
\times \frac{1}{256(2\pi)^{8}E^{3}_{beam}}
\frac{{|\cal M|}^{2}}{2x_{1}x_{2}} \times
\frac{p_{1}^{2}}{E_{1}} \frac{p_{2}^{2}}{E_{2}} 
\frac{p_{\ell}^{2}}{E_{\ell}} \frac{1}{E_{\nu}}
\times W_{Lepton}W_{Jet1}W_{Jet2} \times \Theta_{\rm{Cuts}}(\vec{x})
\end{eqnarray}


%---------------------------------------------------------------------
\section{Introduction}

The search for single top quark production at the Tevatron has turned
out to be a significantly more difficult task than anybody
anticipated. Despite the large production cross section predicted by
the standard model (about half of that for top quark pairs) and the
distinct event signature involving a leptonic $W$ boson decay and two
or more jets (at least one being a $b$~jet), this top quark production
mode has not yet been observed. The main reason is that signal events
are overwhelmed by a background from $W$+jets orders of magnitude
larger. At jet multiplicities above two, even $t\bar{t}$ becomes a
significant background. Both backgrounds can mimic the signal event
characteristics in such a way that no cut on a single distribution can
be used to effectively remove the background while preserving a large
enough signal fraction. However, the combined information from
several discriminant variables in a multivariate approach can
potentially achieve enough sensitivity to unambiguously establish a
signal. By now this is widely acknowledged as a requirement for any
viable search for single top production, and all ongoing analyses at
{\dzero} are of a multivariate nature.

Methods such as neural networks or decision trees are being
succesfully used at D{\O} to build efficient event discriminants to
separate signal from background. These methods rely on identifying an
optimal (and usually very large) set of kinematic and topological
variables that collectively encode most of the available information
to classify signal and background events. These methods are ``learning
machines'' since they adjust to minimize the misclassification rate in
samples of pseudo-data containing pure signal and background events
that are presented to them. In this note we describe the so-called
``Matrix Element-based Single Top Search'' which, as will be explained
below, is intrinsically different in approach.

First of all, the matrix element (ME) method attempts to maximize the
sensitivity to the signal by making exhaustive use of the available
kinematic information in the event, as contained in the four-momenta
for the reconstructed objects. Information regarding $b$~tagging has
also been incorporated in the current version of the analysis. This is
the complete set of variables considered in this analysis. Then,
starting from these four-momenta, this method uses the matrix elements
for the different signal and background processes to numerically
compute the event probability density for each hypothesis (signal and
background). This method has been developed and successfully applied
at {\dzero} in the past for parameter estimation such as the top quark
mass~\cite{Abazov:2004cs,Abazov:2006bd} or the longitudinal $W$~boson
helicity fraction in top quark decays~\cite{Abazov:2004ym}. In this
analysis, the computed event probability densities are used to build
an optimal event discriminant between signal and background,
representing the first application of this method to a search at
{\dzero}.

The analysis described in this note is based on the selected
``lepton+jets'' data samples described in Ref.~\cite{general-note}.
The current analysis is restricted to two-jet and three-jet events,
with at least one of the jets $b$~tagged.

This note is organized as follows. Section~\ref{separation-ME}
presents an overview of the method, from the calculation of the event
probability density functions, to the construction of the different
event discriminants. Section~\ref{data-MC} contains comparisons
between data and the background model for the different event
discriminant distributions from control and signal samples.
Section~\ref{sec:ensembles} presents results of the analysis run on a
number of Monte Carlo ensembles. Section~\ref{exp-performance} is
devoted to a discussion of the expected performance, both in terms of
significance and cross section measurement. Section~\ref{results}
presents the measured results from data, followed by
Section~\ref{sec:eventcharacteristics} which shows characteristics of
the events selected according to their values of the signal
discriminants. Finally, Section~\ref{summary} is devoted to a summary
and conclusions.



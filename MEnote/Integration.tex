 \section{Integration Variable Remapping}

\subsection{Introduction}
This section will layout the jacobian needed for the 10 $\rar$ 10 remapping of
variables for the parton level cross section. The base variables used are shown below.

\begin{myitemize}
\item $p_{3}$: Absolute momentum of the lepton
\item $p_{5}$: Absolute momentum of the first quark
\item $p_{6}$: Absolute momentum of the second quark
\item $p_{tot}^{z}$: Total $p_{z}$ of the system
\item $cos(\theta_{3})$: Cosine($\theta$) of the lepton
\item $\phi_{3}$: $\phi$ of the lepton
\item $cos(\theta_{5})$: Cosine($\theta$) of the first quark
\item $\phi_{5}$: $\phi$ of the first quark
\item $cos(\theta_{6})$: Cosine($\theta$) of the second quark
\item $\phi_{6}$: $\phi$ of the second quark
\end{myitemize}

Other variables that are useful for the integration are
\begin{myitemize}
\item $m_{34}$: Mass of the lepton and neutrino (W mass)
\item $m_{345}$: Mass of the lepton, neutrino, and first quark (top mass)
\item $m_{56}$: Mass of the first and second quark ($b\bar{b}$ mass)
\end{myitemize}

Since some of these variables are sharp peaks (W and top masses), it is much
better to sample from the expected distribution rather than make requirements of
the invariant masses. The W and top masses are expected to follow a Breit-Wigner
distribution shown below.

\begin{equation}
\sigma(M_{34}) = \frac{1}{\pi} \left[ \frac{\gamma}{(M_{34} - M_{W})^{2} +
\gamma^{2}} \right]
\end{equation}

where $M_{34}$ is the mass of the lepton and neutrino. Similarly, the top mass
has the following expected distribution.

\begin{equation}
\sigma(M_{345}) = \frac{1}{\pi} \left[ \frac{\gamma}{(M_{345} - M_{top})^{2} +
\Gamma^{2}} \right]
\end{equation}

where  $M_{345}$ is the mass of the lepton, neutrino, and first quark.

\subsection{Sampling from a Breit-Wigner mass distribution}

Sampling from a Breit-Wigner distribution is done by selecting a random point
between 0 and 1 from the cumulative distribution function of the BW
function. The cumulative distribution function, of cdf, for the Breit-Wigner
distribution is shown below.

\begin{equation}
\label{sample}
\int \sigma(m;m_{0};\Gamma) = F(m;m_{0},\Gamma) = \frac{1}{\pi} \tan^{-1} \left[
\frac{m-m_{0}}{\Gamma} \right] + \frac{1}{2}
\end{equation}

The value of F is taken as a random number between 0 and 1. After selecting a
value of F, the next step is to solve for m. As a function of F, defined as u for
the following, the mass is 


\begin{equation}
m = m_{0} + \Gamma \tan \left[ \pi(u - \frac{1}{2}) \right]
\end{equation}

\subsection{Sampling from a Breit-Wigner $S_{cm}$ distribution}

In the previous example, a distribution was sampled using a random number
uniformly distributed from 0 to 1. In, this example, a new random number is used
that is uniformaly samples from 0 to 1, but the maximum and minimum values of
the variable are taken into account in the Jacobian.

The distribution of the variable $S_{cm}$ is the following

\begin{equation}
\label{defines}
s = m_{0}^{2} + m_{0}\Gamma \tan \left[ m_{0} \Gamma r \right]
\end{equation}

where r is defined in terms of the random variable, u, that is uniformaly
distributed between 0 and 1.

\begin{equation}
\label{rtou}
r = (r_{max} - r_{min}) \times u + r_{min}
\end{equation}

where $r_{max}$ and $r_{min}$ are defined in terms of the variable $s_{cm}$.

\begin{eqnarray}
\label{definer}
r = \frac{1}{m_{0}\Gamma} \tan \left[ \frac{s - m_{0}^{2}}{m_{0}\Gamma} \right] \\
r_{min} = \frac{1}{m_{0}\Gamma} \tan \left[ \frac{s_{min} -
m_{0}^{2}}{m_{0}\Gamma} \right] \\
r_{max} = \frac{1}{m_{0}\Gamma} \tan \left[ \frac{s_{max} - m_{0}^{2}}{m_{0}\Gamma} \right]
\end{eqnarray}



\subsection{Jacobian for random sampling of a Breit-Wigner distribution around
the W mass squared, $s_{34}$}

The first case to consider is the Breit-Wigner sampling around the W mass
$S_{34}$ distribution and replace the integration variable, $p_{3}$ or the
lepton momentum.

\begin{equation}
\label{jacobian1}
|J(p_{3}, u)| = \left| \pderiv{p_{3}}{u} \right| 
\end{equation}

Because u is redined in terms of the variable r, we can rewrite ~\ref{jacobian1}
in terms of r instead of u.

\begin{equation}
\label{jacobian2}
|J(p_{3}, u)| = \left| \pderiv{p_{3}}{u} \right| = \left| \pderiv{p_{3}}{r}
\times \pderiv{r}{u} \right|
\end{equation}

And since the variable r is sampling the $S_{34}$ distribution it makes sense to
define the Jacobian in terms of this variable instead of $p_{3}$.

\begin{equation}
\label{jacobian3}
|J(p_{3}, u)| = \left| \pderiv{p_{3}}{r} \times \pderiv{r}{u} \right| = \left|
\pderiv{p_{3}}{s_{34}} \times \pderiv{s_{34}}{r} \times \pderiv{r}{u} \right| =
\left | \frac{\pderiv{s_{34}}{r} \times
\pderiv{r}{u}}{\pderiv{s_{34}}{p_{3}}} \right |
\end{equation}

Equation ~\ref{jacobian3} has three components: $\pderiv{s_{34}}{r}$,
$\pderiv{r}{u}$, and $\pderiv{s_{34}}{p_{3}}$. From equation ~\ref{rtou}, the
partial derivative of r with respect to u is 

\begin{equation}
\label{drdu}
\pderiv{r}{u} = r_{max} - r_{min} = \Delta r
\end{equation}

Next, the partial of $s_{34}$ with respect to r can be determined from equation
~\ref{defines}.


\begin{equation}
\label{ds34dr}
\pderiv{s_{34}}{r} = (m_{W} \Gamma_{W})^{2} \sec^{2} \left[m_{W} \Gamma_{W} r \right]
\end{equation}

Inserting the value of r(s) as defined in equation ~\ref{definer}, equation
~\ref{ds34dr} can be re-written as

\begin{equation}
\label{ds34dr_2}
\pderiv{s_{34}}{r} = (m_{W} \Gamma_{W})^{2} \sec^{2} \left[m_{W} \Gamma_{W} r \right] =
(m_{W} \Gamma_{W})^{2} \sec^{2} \left[\arctan \left[ \frac{s_{34} - m_{W}^{2}}{m_{W}
\Gamma_{W}}  \right] \right]
\end{equation}

Equation ~\ref{ds34dr_2} is solved by defining a right triangle where

\begin{eqnarray}
\nonumber
\tan(\theta) = \frac{s_{34}-m_{W}^{2}}{m_{W}\Gamma_{W}} \\
\cos(\theta) = \frac{1}{\sqrt{1+\left[ \frac{s_{34}-m_{W}^{2}}{m_{W}\Gamma_{W}}
\right]^{2}}}
\end{eqnarray}

Using these definitions, equation ~\ref{ds34dr_2} is finally defined as

\begin{equation}
\label{ds34dr_3}
\pderiv{s_{34}}{r} = (m_{W} \Gamma_{W})^{2} \sec^{2} \left[\arctan \left[ \frac{s_{34} - m_{W}^{2}}{m_{W}
\Gamma_{W}}  \right] \right] = (m_{W}\Gamma_{W})^{2} + (s_{34} - m_{W}^{2})^{2}
\end{equation}

Finally, we need the partial derivative of $s_{34}$ with repect to
$p_{3}$. First, we define $s_{34}$

\begin{equation}
\label{defines34}
s_{34} = m_{3}^{2} + m_{4}^{2} + 2E_{3}E_{4} - 2p_{3}^{x}p_{4}^{x} - 2p_{3}^{y}p_{4}^{y} - 2p_{3}^{z}p_{4}^{z}
\end{equation}

Since the neutino four-vector is defined in terms of all the other particles in
the event, we need to rewrite equation ~\ref{defines34} to expose all the
dependences on $p_{3}$. For the following, it is assumed that the lepton and
neutrino are massless meaning $E_{3} = p_{3}$. 

\begin{eqnarray}
\label{defines34_2}
\nonumber
s_{34} = 2p_{3}\sqrt{(-p_{3}^{x} -p_{5}^{x} -p_{6}^{x})^{2} + (-p_{3}^{y} -p_{5}^{y} -p_{6}^{y})^{2} + (p_{tot}^{z} -p_{3}^{z} -p_{5}^{z} -p_{6}^{z})^{2}} \\
- 2p_{3}^{x}(-p_{3}^{x} -p_{5}^{x} -p_{6}^{x}) -
2p_{3}^{y}(-p_{3}^{y} -p_{5}^{y} -p_{6}^{y}) - 2p_{3}^{z}(p_{tot}^{z} -p_{3}^{z} -p_{5}^{z} -p_{6}^{z})
\end{eqnarray}

After combining like terms, we can evaluate the partial derivative of $s_{34}$
with respect to $p_{3}$ that yields the relatively
simple formula

\begin{equation}
\label{ds34dp3}
\pderiv{s_{34}}{p_{3}} = 2(p_{3} + p_{4})(1 - \hat{p_{3}} \cdot \hat{p_{4}})
\end{equation}

Finally, we can rewrite the Jacobian defined in ~\ref{jacobian3} as

\begin{equation}
\label{jacobian4}
|J(p_{3}, u)| = \left | \frac{\pderiv{s_{34}}{r} \times
\pderiv{r}{u}}{\pderiv{s_{34}}{p_{3}}} \right | = \frac{\Delta R \times \left[
(m_{W}\Gamma_{W})^{2} + (s_{34} - m_{W}^{2})^{2} \right]}{2(p_{3} + p_{4})(1 - \hat{p_{3}} \cdot \hat{p_{4}})}
\end{equation}

In some cases, it is also common to replace the first quark momentum integration
with the Breit-Wigner sampling variable. In that case, we need to evaluate

\begin{equation}
\label{jacobian_5}
|J(p_{5}, u)| = \left | \frac{\pderiv{s_{34}}{r} \times
\pderiv{r}{u}}{\pderiv{s_{34}}{p_{5}}} \right |
\end{equation}

For this substitution, we only need to evaluate the partial derivative of
$s_{34}$ with respect to $p_{5}$. Assuming a massless quark, the result is

\begin{equation}
\label{ds34dp5}
\pderiv{s_{34}}{p_{5}} = 2p_{3}(\hat{p_{3}} \cdot \hat{p_{5}} - \hat{p_{4}} \cdot \hat{p_{5}})
\end{equation}

Combining equation ~\ref{ds34dp5} with ~\ref{jacobian_5} yields

\begin{equation}
\label{jacobian_6}
|J(p_{5}, u)| = \left | \frac{\pderiv{s_{34}}{r} \times
\pderiv{r}{u}}{\pderiv{s_{34}}{p_{5}}} \right | = \frac{\Delta R \times \left[
(m_{W}\Gamma_{W})^{2} + (s_{34} - m_{W}^{2})^{2} \right]}{2p_{3}(\hat{p_{3}} \cdot \hat{p_{5}} - \hat{p_{4}} \cdot \hat{p_{5}})}
\end{equation}



%
%-----------------------------------------------------------------------
%



\subsection{Jacobian for random sampling of a Breit-Wigner distribution around
the top mass squared, $s_{345}$}

The next case to consider is the Breit-Wigner sampling around the top mass squared
$S_{345}$ distribution and replace the integration variable, $p_{3}$ or the
lepton momentum. As before, we need need to calculate the following

\begin{equation}
\label{jacobian_1}
|J(p_{3}, u)| = \left| \pderiv{p_{3}}{r} \times \pderiv{r}{u} \right| = \left|
\pderiv{p_{3}}{s_{345}} \times \pderiv{s_{345}}{r} \times \pderiv{r}{u} \right| =
\left | \frac{\pderiv{s_{345}}{r} \times
\pderiv{r}{u}}{\pderiv{s_{345}}{p_{3}}} \right |
\end{equation}

Equation ~\ref{jacobian_1} has three components: $\pderiv{s_{345}}{r}$,
$\pderiv{r}{u}$, and $\pderiv{s_{345}}{p_{3}}$. We know $\pderiv{r}{u}$ from
equation ~\ref{drdu} where $r_{min}$ and $r_{max}$ are defined by the $s_{345}$
system instead of the $s_{34}$ system. We also know $\pderiv{s_{345}}{r}$ from ~\ref{ds34dr_3}
where we replace $s_{34}$ with $s_{345}$.

\begin{equation}
\label{ds345dr}
\pderiv{s_{345}}{r} = (m_{t}\Gamma_{t})^{2} + (s_{345} - m_{t}^{2})^{2}
\end{equation}

We do need the partial derivative of $s_{345}$ with repect to
$p_{3}$. First, we define $s_{345}$

\begin{eqnarray}
\label{defines345}
\nonumber
s_{345} = m_{3}^{2} + m_{4}^{2} + m_{5}^{2} + 2E_{3}E_{4} + 2E_{3}E_{5} +
2E_{4}E_{5} - \\
2p_{3}^{x}p_{4}^{x} - 2p_{3}^{x}p_{5}^{x} - 2p_{4}^{x}p_{5}^{x} -
2p_{3}^{y}p_{4}^{y} - 2p_{3}^{y}p_{5}^{y} - 2p_{4}^{y}p_{5}^{y} -
2p_{3}^{z}p_{4}^{z} - 2p_{3}^{z}p_{5}^{z} - 2p_{4}^{z}p_{5}^{z}
\end{eqnarray}

As before, the neutino four-vector is defined in terms of all the other particles in
the event so we need to rewrite equation ~\ref{defines345} to expose all the
dependences on $p_{3}$. For the following, it is assumed that the lepton,
neutrino, and quark are massless meaning $E_{3} = p_{3}$.

\begin{eqnarray}
\label{defines345_2}
\nonumber
s_{345} = 2p_{3}\sqrt{(-p_{3}^{x} -p_{5}^{x} -p_{6}^{x})^{2} + (-p_{3}^{y}
-p_{5}^{y} -p_{6}^{y})^{2} + (p_{tot}^{z} -p_{3}^{z} -p_{5}^{z} -p_{6}^{z})^{2}}
+ \\
\nonumber
2p_{3}p_{5} + 2\sqrt{(-p_{3}^{x} -p_{5}^{x} -p_{6}^{x})^{2} + (-p_{3}^{y} -p_{5}^{y}
-p_{6}^{y})^{2} + (p_{tot}^{z} -p_{3}^{z} -p_{5}^{z} -p_{6}^{z})^{2}}p_{5} - \\
\nonumber
2p_{3}^{x}(-p_{3}^{x} -p_{5}^{x} -p_{6}^{x}) - 2p_{3}^{x}p_{5}^{x} -
2(-p_{3}^{x} -p_{5}^{x} -p_{6}^{x})p_{5}^{x} - \\
\nonumber
2p_{3}^{y}(-p_{3}^{y} -p_{5}^{y} -p_{6}^{y}) - 2p_{3}^{y}p_{5}^{y} -
2(-p_{3}^{y} -p_{5}^{y} -p_{6}^{y})p_{5}^{y} - \\
2p_{3}^{z}(-p_{3}^{z} -p_{5}^{z} -p_{6}^{z}) - 2p_{3}^{z}p_{5}^{z} - 2(-p_{3}^{z} -p_{5}^{z} -p_{6}^{z})p_{5}^{z}
\end{eqnarray}

After combining like terms, we can evaluate the partial derivative of $s_{345}$
with respect to $p_{3}$ as

\begin{equation}
\label{ds345dp3}
\pderiv{s_{345}}{p_{3}} = 2(p_{3} + p_{4} + p_{5})(1 - \hat{p_{3}} \cdot \hat{p_{4}})
\end{equation}

Finally, we can rewrite the Jacobian defined in ~\ref{jacobian_1} as

\begin{equation}
\label{jacobian_2}
|J(p_{3}, u)| = \left | \frac{\pderiv{s_{345}}{r} \times
\pderiv{r}{u}}{\pderiv{s_{345}}{p_{3}}} \right | = \frac{\Delta R \times \left[
(m_{t}\Gamma_{t})^{2} + (s_{345} - m_{t}^{2})^{2} \right]}{2(p_{3} + p_{4} + p_{5})(1 - \hat{p_{3}} \cdot \hat{p_{4}})}
\end{equation}


Instead of replacing the lepton momentum integration variable, it is also common
to replace the first quark momentum integration variable, $p_{5}$. In that case,
we need to evaluate the following Jacobian.

\begin{equation}
\label{jacobian_3}
|J(p_{5}, u)| = \left| \pderiv{p_{5}}{r} \times \pderiv{r}{u} \right| = \left|
\pderiv{p_{5}}{s_{345}} \times \pderiv{s_{345}}{r} \times \pderiv{r}{u} \right| =
\left | \frac{\pderiv{s_{345}}{r} \times
\pderiv{r}{u}}{\pderiv{s_{345}}{p_{5}}} \right |
\end{equation}

The only difference is that partial derivative of $s_{345}$ with respect to
$p_{5}$ instead of $p_{3}$. However, since $s_{345}$ is invariant under a change
of $p_{3}$ and $p_{5}$ the partial derivatives must be equal. Thus, 

\begin{equation}
\label{jacobian_4}
|J(p_{5}, u)| = \left | \frac{\pderiv{s_{345}}{r} \times
\pderiv{r}{u}}{\pderiv{s_{345}}{p_{5}}} \right | = \frac{\Delta R \times \left[
(m_{t}\Gamma_{t})^{2} + (s_{345} - m_{t}^{2})^{2} \right]}{2(p_{3} + p_{4} + p_{5})(1 - \hat{p_{4}} \cdot \hat{p_{5}})}
\end{equation}





%
%-----------------------------------------------------------------------
%


\subsection{Jacobian for random sampling of two Breit-Wigner distributions around
the top mass squared, $s_{345}$ and W mass squared, $s_{34}$}


The next situation is to sample from a Breit-Wigner around the
top mass squared and the W mass squared, or $s_{345}$ and $s_{34}$. It is common
to replace the lepton momentum and first quark momentum integration variables
with the two new variables. Since we are
replacing two variables, we need to evaluate the following Jacobian

\begin{equation}
|J(p_{3}, p_{5} ; u_{1}, u_{2})| = \left| \begin{array}{cc}
\pderiv{p_{3}}{u_{1}}	& \pderiv{p_{3}}{u_{2}} \\
\pderiv{p_{5}}{u_{1}}	& \pderiv{p_{5}}{u_{2}} \\
\end{array} \right|
\end{equation}

where $u_{1}$ and $u_{2}$ are the sampling variables around the top mass squared
and W mass squared, respectively.

We have already computed the partial derivatives for each of these cases
in the previous two sections, thus the result is

\begin{eqnarray}
\nonumber
|J(p_{3}, p_{5} ; u_{1}, u_{2})| = \left| \begin{array}{cc}
\pderiv{p_{3}}{u_{1}}	& \pderiv{p_{3}}{u_{2}} \\
\pderiv{p_{5}}{u_{1}}	& \pderiv{p_{5}}{u_{2}} \\
\end{array} \right| = 
\nonumber
\left| \begin{array}{cc}
\frac{\pderiv{s_{345}}{r} \times \pderiv{r}{u}}{\pderiv{s_{345}}{p_{3}}}	& \frac{\pderiv{s_{345}}{r} \times
\pderiv{r}{u}}{\pderiv{s_{345}}{p_{5}}} \\
\frac{\pderiv{s_{34}}{r} \times \pderiv{r}{u}}{\pderiv{s_{34}}{p_{3}}}	& \frac{\pderiv{s_{34}}{r} \times
\pderiv{r}{u}}{\pderiv{s_{34}}{p_{5}}}
\end{array} \right| = \\
\left| \begin{array}{cc}
\frac{\Delta R_{345} \times \left[
(m_{t}\Gamma_{t})^{2} + (s_{345} - m_{t}^{2})^{2} \right]}{2(p_{3} + p_{4} + p_{5})(1 - \hat{p_{3}} \cdot \hat{p_{4}})}	& \frac{\Delta R_{34} \times \left[
(m_{W}\Gamma_{W})^{2} + (s_{34} - m_{W}^{2})^{2} \right]}{2(p_{3} + p_{4})(1 -
\hat{p_{3}} \cdot \hat{p_{4}})} \\
\frac{\Delta R_{345} \times \left[
(m_{t}\Gamma_{t})^{2} + (s_{345} - m_{t}^{2})^{2} \right]}{2(p_{3} + p_{4} + p_{5})(1 - \hat{p_{4}} \cdot \hat{p_{5}})}	& \frac{\Delta R_{34} \times \left[
(m_{W}\Gamma_{W})^{2} + (s_{34} - m_{0}^{2})^{2} \right]}{2p_{3}(\hat{p_{3}} \cdot \hat{p_{5}} - \hat{p_{4}} \cdot \hat{p_{5}})}
\end{array}
\right|
\end{eqnarray}


\subsection{Sampling from a Polynomial $S_{cm}$ distribution}

*** This is where I am taking a function from Aurelio and I can't seem to derive
it on my own ***

The distribution of the variable $S_{cm}$ according to a polynomial power
distribution is

\begin{equation}
\label{defines_power}
s = m_{0}^{2} + \left[ (1-\alpha)r \right]^{\frac{-1}{\alpha - 1}}
\end{equation}

where r is defined in terms of the random variable, u, that is uniformaly
distributed between 0 and 1.

\begin{equation}
\label{rtou_power}
r = (r_{max} - r_{min}) \times u + r_{min}
\end{equation}

where $r_{max}$ and $r_{min}$ are defined in terms of the variable $s_{cm}$.

\begin{eqnarray}
\label{definer_power}
r = \frac{1}{1-\alpha} \times \left[ s - m_{0}^{2} \right]^{1-\alpha} \\
r_{min} = \frac{1}{1-\alpha} \times \left[ s_{min} - m_{0}^{2}
\right]^{1-\alpha} \\
r_{max} = \frac{1}{1-\alpha} \times \left[ s_{max} - m_{0}^{2}
\right]^{1-\alpha}
\end{eqnarray}

where alpha can not equal 1.


\subsection{Jacobian for random sampling of a polynomial distribution
starting at $m_{pole}$}

The first case to consider is sampling around a falling polynomial distribution
for the mass squared of two quarks in the event, $s_{56}$. We need to define the
Jacobian with respect $p_{5}$ or $p_{6}$. Since $s_{56}$ is invariant under an
interchange of particle 5 and 6, the Jacobian will be the same for each momentum
integration. The following assume $p_{5}$ will be replaced with the variable,
$u$, which is sampled from a polynomial distribution.

\begin{equation}
\label{jacobian_7}
|J(p_{5}, u)| = \left| \pderiv{p_{5}}{u} \right| 
\end{equation}

Because u is redined in terms of the variable r, we can rewrite ~\ref{jacobian_7}
in terms of r instead of u.

\begin{equation}
\label{jacobian_8}
|J(p_{5}, u)| = \left| \pderiv{p_{5}}{u} \right| = \left| \pderiv{p_{5}}{r}
\times \pderiv{r}{u} \right|
\end{equation}

And since the variable r is sampling the $S_{56}$ distribution it makes sense to
define the Jacobian in terms of this variable instead of $p_{5}$.

\begin{equation}
\label{jacobian_9}
|J(p_{5}, u)| = \left| \pderiv{p_{5}}{r} \times \pderiv{r}{u} \right| = \left|
\pderiv{p_{5}}{s_{56}} \times \pderiv{s_{56}}{r} \times \pderiv{r}{u} \right| =
\left | \frac{\pderiv{s_{56}}{r} \times
\pderiv{r}{u}}{\pderiv{s_{56}}{p_{5}}} \right |
\end{equation}

Equation ~\ref{jacobian_9} has three components: $\pderiv{s_{56}}{r}$,
$\pderiv{r}{u}$, and $\pderiv{s_{56}}{p_{5}}$. From equation ~\ref{rtou_power}, the
partial derivative of r with respect to u is 

\begin{equation}
\label{drdu_power}
\pderiv{r}{u} = r_{max} - r_{min} = \Delta R_{56}
\end{equation}

Next, the partial of $s_{56}$ with respect to r can be determined from equation ~\ref{defines_power}.

\begin{equation}
\label{ds56dr_power}
\pderiv{s_{56}}{r} = \left[ r(1-\alpha) \right]^{\frac{\alpha}{1-\alpha}}
\end{equation}

Inserting the value of r(s) as defined in equation ~\ref{definer_power}, equation ~\ref{ds56dr_power} can be re-written as

\begin{equation}
\label{ds56dr_power2}
\pderiv{s_{56}}{r} = \left[ s_{56} - m_{0}^{2} \right]^{\alpha}
\end{equation}

Finally, we need the partial derivative of $s_{56}$ with repect to $p_{5}$. First, we define $s_{56}$

\begin{equation}
\label{defines56_power}
s_{56} = m_{5}^{2} + m_{6}^{2} + 2E_{5}E_{6} - 2p_{5}^{x}p_{6}^{x} - 2p_{5}^{y}p_{6}^{y} - 2p_{5}^{z}p_{6}^{z}
\end{equation}

we can evaluate the partial derivative of $s_{56}$ with respect to $p_{5}$.

\begin{equation}
\label{ds56dp5_power}
\pderiv{s_{56}}{p_{5}} = 2p_{6}(1 - \hat{p_{5}} \cdot \hat{p_{6}})
\end{equation}

Finally, we can rewrite the Jacobian defined in ~\ref{jacobian_9} as

\begin{equation}
\label{jacobian_10}
|J(p_{5}, u)| = \left | \frac{\pderiv{s_{56}}{r} \times
\pderiv{r}{u}}{\pderiv{s_{56}}{p_{5}}} \right | = \frac{\Delta R_{56} \times \left[ s_{56}-m_{0}^{2} \right]^{\alpha}}{2p_{6}(1 - \hat{p_{5}} \cdot \hat{p_{6}})}
\end{equation}

and 

\begin{equation}
\label{jacobian_11}
|J(p_{6}, u)| = \left | \frac{\pderiv{s_{56}}{r} \times
\pderiv{r}{u}}{\pderiv{s_{56}}{p_{6}}} \right | = \frac{\Delta R_{56} \times \left[ s_{56}-m_{0}^{2} \right]^{\alpha}}{2p_{5}(1 - \hat{p_{5}} \cdot \hat{p_{6}})}
\end{equation}

%---------------------------------------------------------------------
%

\appendix
\section*{Appendix 9 --- Discriminant Optimization Criteria}
\label{appendix-optimization}

Since the goal of the analysis is to validate the production of single
top quarks, any optimization criteria should be based on affirming the
signal+background hypothesis convincingly. We, therefore, define an
optimal search as one that maximizes the probability of the
signal+background hypothesis relative to that of the background
hypothesis. Since we use a Bayesian approach for all our statistical
analyses, a natural way to do this would be to compute the Bayes
factor which is the ratio of the posterior probability densities of
the signal+background hypothesis to the background-only
hypothesis. The description of the method we use to obtain a posterior
probability density can be found in Appendix 13 of
Ref.~\cite{run2-d0-230} and in Ref.~\cite{d0note5123}.

A fundamental principle of Bayesian reasoning is that one should
integrate over all that is unknown.  In designing an analysis, we do
not yet know the event count $n$, nor do we know the background and
signal means $b$ and $s$, respectively. Therefore, in accordance with
the basic Bayesian principle, we must integrate over $n$, $s$, and $b$,
constrained by whatever knowledge we have of them.

We are working to develop the methodology to compute Bayes factor as
discussed above by averaging over all possible value of $n$, $s$ and
$b$~\cite{d0note5260}. For now we have used an approximation to the
Bayes factor by considering only one value of $n$ which is set to the
estimated values of the signal and background yields. The factor is
then computed as illustrated in Fig.~\ref{fig:bayesfactor} using the
values of the posterior probability densities at peak and at zero
signal cross section.

\begin{figure}[!h!tbp]
\includegraphics[width=0.5\textwidth]{figures/bayesfactor.eps}
\vspace{-0.1in}
\begin{minipage}{4in}
\caption[bayesfactor]{A plot of the posterior probability density
versus signal cross section that illustrates the calculation of the
Bayes factor.}
\label{fig:bayesfactor}
\end{minipage}
\end{figure}

We have chosen to use the Bayes ratio to optimize the discriminants'
performance. It may be noted however that this approximate definition
may yield unreasonably high values as the posterior density at zero
signal cross section becomes infinitesimally small. In such a case the
distribution of the posterior probability density is expected to be a
near-perfect Gaussian, and a more meaningful quantity for any
optimization would be a ratio of the signal cross section at the peak
of the posterior to the width of the posterior as shown in
Fig.~\ref{fig:measurement}. This essentially defines how significant a
measurement of the single top cross section would be if the data were
consistent with the signal+background hypothesis.

\begin{figure}[!h!tbp]
\includegraphics[width=0.5\textwidth]{figures/measurement.eps}
\vspace{-0.1in}
\begin{minipage}{4in}
\caption[measurement]{A plot of the posterior probability density
versus signal cross section that illustrates the measurement of the
cross section and its uncertainty.}
\label{fig:measurement}
\end{minipage}
\end{figure}



%---------------------------------------------------------------------
%  Writeup of the improved search of Run II data for single top quark
%  production at DZero.
%  Started: Oct 2006
%  Authors: The Single Top Working Group
%---------------------------------------------------------------------
%

\section{Systematic Uncertainties}
\label{systematics}

Systematic uncertainties enter the single top calculations in two
ways: as uncertainty on the normalization of the background samples;
and as effects that change the shapes of the distributions of those
samples and the shapes of the expected signal distributions.

Table~\ref{tab:generalsys} summarises the relative uncertainties on
each of the sources described below. Detailed tables of uncertainties
for each individual channel are listed in Appendix~6.

We have considered the following systematic uncertainties in the
analysis:

\begin{itemize}
\item {\bf Integrated luminosity} \\ 
The error on the luminosity estimate affects the signal and $\ttbar$
yields.

\item {\bf Theoretical cross sections} \\ 
The uncertainty on the cross section for signal and $\ttbar$ includes
the theoretical error and the uncertainty from the top quark mass
uncertainty.

\item {\bf Trigger efficiency} \\
The trigger efficiency as a function of $\pt$ for each object in an MC
event is shifted up and down by one standard deviation and the weight
of the event recalculated. This shift is taken as an overall constant
systematic error on the MC acceptance.

\item {\bf Primary vertex selection efficiency} \\
The primary vertex selection efficiency in data and MC are not the
same. We assign a systematic uncertainty for the difference on the $z$
position of the primary vertex taking into account the beam profile
along the longitudinal direction~\cite{beamshifts}.

\item {\bf Jet reconstruction and identification} \\
The dependence of the efficiency on the number and the $\eta$ of the
jets is taken into account.

\item {\bf Jet energy scale and jet energy resolution} \\
The JES correction is raised and lowered by one standard deviation and
the whole analysis repeated. In the data, the JES uncertaintiy
contains the jet energy resolution uncertainty. But in the Monte Carlo
the jet energy resolution uncertainty is not taken into account in the
JES uncertainty. To account for this, the Monte Carlo energy smearing
is varied by the size of the jet energy resolution in MC. This
uncertainty affects the acceptance and the shapes of the
distributions.

\item {\bf Jet fragmentation} \\
This systematic error covers the difference in the jet fragmentation
models of {\sc pythia} and {\sc herwig} as well as the uncertainty in
the modeling of initial-state and final-state radiation.

\item {\bf Electron reconstruction and identification efficiency} \\
The electron reconstruction and identification scale factor is
parametrized as a function of the distance between the electron and
the closest jet. The dependences on $\phi$ and $\pt$ of the electron
are included in the systematic error and also the limited statistics
in each bin of the parametrization.

\item {\bf Electron track matching and likelihood efficiency} \\
These scale factors are parametrized as a function of $\eta$ and
$\phi$, and the limited statistics of each bin, as well as the
dependence on the number of jets are taken as the total systematic
uncertainty.

\item {\bf Muon reconstruction and identification efficiency} \\
The MC scale factor uncertainties are estimated by the muon ID group
as coming from the tag/probe method, background subtraction, and
limited statistics in the parametrization.

\item {\bf Muon track matching and isolation} \\
The muon tracking uncertainty, estimated by the muon ID group,
includes uncertainties from the tag/probe method, background
substraction, luminosity and timing bias, and averaging over $\phi$
and the limited statistics in each bin of the scale factor. The muon
isolation uncertainty was estimated based on the scale factor
dependence with the number of jets; and covers the dependences not
taken into account like $\pt$ and $\eta$.

\item {\bf Matrix method normalization} \\
The determination of the number of real-lepton events in data is
affected by the uncertainties associated with the determination of the
probabilities for a loose lepton to be (mis)identified as a (fake)
real lepton, $\varepsilon_{\rm fake{\mbox{\small -}}{\rm lepton}}$ and
$\varepsilon_{\rm real{\mbox{\small -}}{\rm lepton}}$. It is also
affected by the limited statistics of the data sample. See
Ref.~\cite{matrix-method} for more details.

\item {\bf Heavy flavor ratio} \\
The error on the scale factor we apply to set the $Wbb$ and $Wcc$
contributions in the $W$+jets sample, as described in Appendix~3, is
estimated to cover several effects: dependence on the $b$-quark $\pt$,
the difference between the zero tag samples where it is estimated and the
signal samples where it is used, and the intrinsic uncertainty on the value of
the LO cross section it's being applied on. 

\item {\bf Monte Carlo tag-rate functions}\\
The uncertainty associated with the tag-rate functions is evaluated by
raising and lowering the tag rate by one standard deviation for both
the taggability and the tag rate components and determining the new
event tagging weight. The TRF uncertainties originate from several
sources: statistical errors of MC event sets; the assummed fraction of
heavy flavor in the MC QCD for the mistag rate determination; and the
parametrizations.
\end{itemize}

\vspace{-0.1in}
\begin{table}[h]
\begin{center}
\begin{minipage}{5 in}
\begin{ruledtabular}
\begin{tabular}{l|c||l|c}
\multicolumn{4}{c}{\underline{Relative Systematic Uncertainties}}\\
\hline
{\ttbar} cross section~~~~~~~~~~~~ & $18\%$  & Primary vertex                    &  $3\%$  \\
Luminosity                         &  $6\%$  & Electron reco * ID                &  $2\%$  \\
Electron trigger                   & $3\%$   & Electron trackmatch \& likelihood &  $5\%$  \\
Muon trigger                       & $6\%$   & Muon reco * ID                    &  $7\%$  \\
Jet energy scale                   &wide range&Muon trackmatch \& isolation      &  $2\%$  \\
Jet efficiency                     &  $2\%$  & $\varepsilon_{{\rm real-}e}$      &  $2\%$  \\
Jet fragmentation                  & 5--7$\%$& $\varepsilon_{{\rm real-}\mu}$    &  $2\%$  \\
Heavy flavor ratio                 & $30\%$  & $\varepsilon_{{\rm fake-}e}$      &3--40$\%$\\ 
Tag-rate functions                 &2--16$\%$& $\varepsilon_{{\rm fake-}\mu}$    &2--15$\%$
\end{tabular}
\end{ruledtabular}
\caption[gensys]{A summary of the relative systematic uncertainties
for each of the applied corrections and efficiencies. The uncertainty
shown is the error on the correction or the efficiency, before it has
been applied to the MC or data samples.}
\label{tab:generalsys}
\end{minipage}
\end{center}
\end{table}


%  email from Lisa
% ======================================
% Muon channel
% ==============

% 1. Muon ID SF
% (a) Systematic uncertainty estimated by muonID group - 0.7%.
% It includes uncertainties from tag/probe method, background
% subtraction, binning
% (b) Systematic uncertainty coming from the limited statistics
% in each bin of 2D histogram - 7%.
% Second uncertainty is overestimated because we treat errors in
% each bin as fully correlated.
% But we prefer to be concervative for now (if it does not kills
% the analysis, of course) because we do not have a better way to
% estimate it.
% Total - 7%

% 2. Muon tracking SF
% (a) Systematic uncertainty estimated by muonID group - 0.7%.
% It includes uncertainties from tag/probe method, background
% subtraction, binning, lumi and time bias, averaging over phi.
% (b) Systematic uncertainty coming from the limited statistics
% in each bin of 1D histogram - 1.3%.
% Total - 1.5%

% 3. Muon isolation SF
% This systematic uncertainty was not estimated by muonID group.
% We estimated it from the plot showing dependence of isolation SF
% vs # of jets (this is the spc file we are using) and it is 2%.
% We also double checked that 2% covers the dependences we are not
% taking into accout (vs eta and pt) which are available in muon
% certification note.

% Total systematics from all sources on the muon: 7.4% (was 5.1%
% in p14).


% =========================================
% Electron channel
% ==============
% The estimate of the systematic uncertainties from EMID group is
% not yet available.

% 1. EM preselection SF (includes emid 10,11, isolation, emfraction)
% (a) Systematic uncertainty coming from the dependencies we do not
% take into account (phi, pt) - 2%
% (b) Systematic uncertainty coming from the limited statistics in
% each bin of 1D histogram - 1%.
% Total - 2.2%

% 2. EM postselection SF (includes hmatrix, trak-match and LH)
% (a) Systematic uncertainty coming from the dependence on the
% number of jets which was not taken into account - 4%
% (b) Systematic uncertainty coming from the limited statistics
% in each bin of 2D histogram - 3%.
% Total - 5%

% Total systematics from all sources on the electron: 5.5% (was 3.2%
% in p14).


% ========================================
% Primary vertex
% ==================
% From D0 note 5142 and p14 SFs we extracted the following
% uncertainties on PV SF and vertex z position simulation:
% electron channel - 2.4%
% muon channel - 3%

% ========================================
% Uncertainty on the jet reconstruction efficiency SF
% =====================================
% Amnon has made a back of the envelope propagation of the new
% uncertainties he has just released on the ttbar efficiency and
% obtained 1.8% uncertainty. This number depends on the # of jets
% and the distribution of the jets vs eta. Given that single top
% has 2 or 3 jets but they are more forward (ICD and EC regions
% have higher uncrrtainties) I obtained average uncertainty for
% single top of 1.5%.


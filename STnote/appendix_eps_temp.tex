%---------------------------------------------------------------------
%  Writeup of the improved search of Run II data for single top quark
%  production at DZero.
%  Started: Oct 2006
%  Authors: The Single Top Working Group
%---------------------------------------------------------------------
%

\appendix
\section*{Appendix 2 --- Lepton Isolation Probabilities}
\label{epsilons}

Normalization of the $W$+jets and multijet backgrounds to data is
performed using the matrix method as explained in
Section~\ref{matrix-method}. To determine the separate contributions
from signal-like real-lepton events and background fake-lepton events
in our data, we rely on the measurement of two probabilities,
$\varepsilon_{{\rm real-}\ell}$ and $\varepsilon_{{\rm fake-}\ell}$.

$\varepsilon_{{\rm real-}\ell}$ is the probability for a real lepton
to pass the isolation requirements. It is obtained using a $Z{\rar}
ee$ sample for the electrons and $Z {\rar} \mu\mu$ sample for the
muons, one of the leptons is ``tagged'' as \emph{tight}, then the
$\varepsilon_{{\rm real-}\ell}$ is calculated measuring the
probability of the other ``probe'' lepton to pass the
\emph{tight} selection cut.

The $\pt$ and $\eta$ dependence for the electron case is shown in
Fig.~\ref{cc-eps-signal}. The $\pt$ and jet multiplicity dependence in
the muon case is shown in Fig.~\ref{mu-eps-signal}. The average values
for $\varepsilon_{{\rm fake-}e}$ and $\varepsilon_{{\rm fake-}\mu}$ as
a function of the jet multiplicity are given in
Table~\ref{eps-signal}. The overall uncertainty for $\varepsilon_{{\rm
fake-}\ell}$ is $1\%$ statistical and $2\%$ systematic.

%These numbers are from the ttbar group, our average number for
%electrons are \[\varepsilon_{{\rm real-}\ell}(electrons)
%=83.8\pm1.0(stat)\pm2.2(sys)\%\] \[\varepsilon_{{\rm real-}\ell}(muons)
%=84.9\pm1.0(stat)\pm2.0(sys)\%\]
%The table contains the average value for each jet, I don't have the
%information about the error

\begin{figure}[!h!tbp]
\includegraphics[width=0.4\textwidth]
{figures/eps_sig_pt_electron.eps}
\hspace{0.5in}
\includegraphics[width=0.4\textwidth]
{figures/eps_sig_eta_electron.eps}
\caption[ccepssignal]{Transverse momentum and detector
pseudorapidity dependence of $\varepsilon_{{\rm real-}e}$.}
\label{cc-eps-signal}
\end{figure} 

\vspace{-0.1in}
\begin{figure}[!h!tbp]
\includegraphics[width=0.4\textwidth]
{figures/mu_eps_sig_toploose_data_pt.eps}
\hspace{0.5in}
\includegraphics[width=0.4\textwidth]
{figures/mu_eps_sig_toploose_data_njets.eps}
\vspace{-0.1in}
\caption[muepssignal]{Transverse momentum and jet
multiplicity dependence of $\varepsilon_{{\rm real-}\mu}$.}
\label{mu-eps-signal}
\end{figure} 

\clearpage

\begin{table}[!h!tbp]
\begin{center}
\begin{minipage}{2.2 in}
\begin{ruledtabular}
\begin{tabular}{c||cc}
\multicolumn{3}{c}{\underline{Real-Lepton Probabilities}}
\vspace{0.1in}\\
No. of jets & $\varepsilon_{{\rm real-}e}$
            & $\varepsilon_{{\rm real-}\mu}$ \\
\hline
  1  &  $87.3\%$  &  $99.1\%$  \\
  2  &  $87.4\%$  &  $98.9\%$  \\
  3  &  $87.4\%$  &  $98.7\%$  \\
  4  &  $87.5\%$  &  $96.1\%$
\end{tabular}
\end{ruledtabular}
\vspace{-0.1in}
\caption[epssignal]{Average values for $\varepsilon_{{\rm real-}\ell}$
in the electron and muon samples for different jet multiplicities.}
\label{eps-signal}
\end{minipage}
\end{center}
\end{table}

$\varepsilon_{{\rm fake-}\ell}$ is the probability for a fake lepton
to pass the isolation requirements. It is obtained using a data
sample. Assuming that the low $\met$ region ($\met < 10$~GeV) is
dominated by the multijet background, and that $\varepsilon_{{\rm
fake-}\ell}$ is independent of the $\met$ cut, then $\varepsilon_{{\rm
fake-}\ell}$ is determined as the ratio of \emph{tight} over
\emph{loose} events in the low $\met$ region. We studied
$\varepsilon_{{\rm fake-}\ell}$ as a function of several parameters,
finding no dependence on the lepton $\pt$ or lepton $\eta$, but
dependence on the trigger version and jet multiplicity. A summary of
the results is presented in Table~\ref{eps-qcd}.

\vspace{0.2in}
\begin{table}[!h!tbp]
\begin{center}
\begin{minipage}{6 in}
\begin{ruledtabular}
\begin{tabular}{c||ccccc|c}
\multicolumn{7}{c}{\hspace{1in}\underline{Fake-Lepton Probabilities}}
\vspace{0.1in}\\
& \multicolumn{6}{c}{Trigger Version}\\
No. of jets & v8--v11 & v12 & v13a & v13b & v14 & Combined\\
\hline 
\multicolumn{7}{l}{Electron channel, $\varepsilon_{{\rm fake-}e}$} \\
 1 & $(11.2 \pm 0.5)\%$ & $(17.9 \pm 0.6)\%$ & $(18.7 \pm 1.2)\%$ & $(19.1 \pm 0.6)\%$ & $(18.5 \pm 0.6)\%$ & $(16.8 \pm 0.3)\%$ \\
 2 & $(12.8 \pm 1.0)\%$ & $(19.2 \pm 1.0)\%$ & $(18.8 \pm 2.2)\%$ & $(19.4 \pm 1.1)\%$ & $(22.0 \pm 1.2)\%$ & $(18.8 \pm 0.5)\%$ \\
 3 & $(13.6 \pm 1.5)\%$ & $(19.5 \pm 1.6)\%$ & $(19.8 \pm 3.4)\%$ & $(19.2 \pm 1.6)\%$ & $(19.4 \pm 1.7)\%$ & $(18.5 \pm 0.8)\%$ \\
 4 & $(10.0 \pm 2.8)\%$ & $(15.5 \pm 2.9)\%$ & $(20.9 \pm 8.6)\%$ & $(17.7 \pm 3.3)\%$ & $(20.7 \pm 3.7)\%$ & $(16.5 \pm 1.6)\%$ \\
\multicolumn{7}{l}{Muon channel, $\varepsilon_{{\rm fake-}\mu}$} \\
 1 & $(11.2 \pm 0.5)\%$ & $(17.9 \pm 0.6)\%$ & $(18.7 \pm 1.2)\%$ & $(19.1 \pm 0.6)\%$ & $(18.5 \pm 0.6)\%$ & $(46.6 \pm 5.4)\%$ \\
 2 & $(12.8 \pm 1.0)\%$ & $(19.2 \pm 1.0)\%$ & $(18.8 \pm 2.2)\%$ & $(19.4 \pm 1.1)\%$ & $(22.0 \pm 1.2)\%$ & $(44.6 \pm 3.4)\%$ \\
 3 & $(13.6 \pm 1.5)\%$ & $(19.5 \pm 1.6)\%$ & $(19.8 \pm 3.4)\%$ & $(19.2 \pm 1.6)\%$ & $(19.4 \pm 1.7)\%$ & $(42.3 \pm 3.8)\%$ \\
 4 & $(10.0 \pm 2.8)\%$ & $(15.5 \pm 2.9)\%$ & $(20.9 \pm 8.6)\%$ & $(17.7 \pm 3.3)\%$ & $(20.7 \pm 3.7)\%$ & $(47.2 \pm 7.7)\%$
\end{tabular}
\end{ruledtabular}
\vspace{-0.1in}
\caption[epsqcd]{$\varepsilon_{{\rm fake-}e}$ and
$\varepsilon_{{\rm fake-}\mu}$ as a function of the trigger version
and jet multiplicity.}
\label{eps-qcd}
\end{minipage}
\end{center}
\end{table}

\vspace{0.1in}
Figures~\ref{electron-eps-qcd} and \ref{muon-eps-qcd}
show the $\met$ dependence for different jet multiplicities and all
trigger versions combined.

\clearpage

\begin{figure}[!h!tbp]
\includegraphics[width=0.55\textwidth]
{figures/epsilonqcd_electron_alltrigs}
\vspace{-0.1in}
\caption[elepsqcd]{Ratio of tight to loose electron events as a
function of $\met$, for different jet multiplicities ($N-{\rm
jets}=1,2,3,4$) and all the trigger versions combined. A fit is shown
in the low $\met$ region, representing the $\varepsilon_{{\rm
fake-}e}$ for that particular jet multiplicity and all triggers.}
\label{electron-eps-qcd}
\end{figure}

\begin{figure}[!h!tbp]
\includegraphics[width=0.55\textwidth]
{figures/epsilonqcd_muon_alltrigs}
\vspace{-0.1in}
\caption[muepsqcd]{Ratio of tight to loose muon events as a
function of $\met$, for different jet multiplicities ($N_{\rm
jets}=1,2,3,4$) and all the trigger versions combined. A fit is shown
in the low $\met$ region, representing the $\varepsilon_{{\rm
fake-}\mu}$ for that particular jet multiplicity and all triggers.}
\label{muon-eps-qcd}
\end{figure}

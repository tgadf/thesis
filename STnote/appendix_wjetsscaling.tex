%---------------------------------------------------------------------
%  Writeup of the improved search of Run II data for single top quark
%  production at DZero.
%  Started: Oct 2006
%  Authors: The Single Top Working Group
%---------------------------------------------------------------------
%
\appendix
\section*{Appendix 3 --- Scaling the $W$+Jets Heavy Flavor Fraction}
\label{appendix-wjets-scaling}

The {\sc alpgen} leading order cross section calculations for
$Wb\bar{b}$, $Wc\bar{c}$, and $Wjj$ have sizeable uncertainties and
are very sensitive to the renormalization and factorization scales. In
previous analyses, we used MCFM~\cite{MCFM} with the same {\sc alpgen}
parton level cuts to estimate the NLO cross sections of $Wb\bar{b}$
and $Wjj$, set the relative fraction of $Wb\bar{b}$ and $Wjj$ based on
these numbers, and then normalized the overall $W$+jets sample to data
based on the matrix method results. That gave us the best estimate,
with uncertaintiess, of what fraction of events in the $W$+jets sample
had heavy flavor jets. With the new matched {\sc alpgen} samples, {\sc
alpgen} is actually run with no parton level cuts since the matching
scheme takes care of avoiding divergences by clustering partons into
parton-jets with certain cuts ($\pt>8$~GeV and $\Delta R>0.4$). For a
description of the matching scheme, and the details of our production
samples, see Ref.~\cite{FOMEcs}. Since MCFM and matched {\sc alpgen}
do not agree in the LO cross section, it is difficult to use the MCFM
NLO cross section to scale our {\sc alpgen} generated samples.

As done in previous analyses, we scale the total $W$+jets leading
order yield to data before $b$-tagging by means of the ``matrix
method,'' as explained in detail in Section~\ref{matrix-method}.
Table~\ref{mmwjets-factors} presents the scale factors from the matrix
method needed to make the LO {\alpgen} $W$+jets yields match the
data. This normalization is done for $Wb\bar{b}$, $Wc\bar{c}$, and
$Wjj$ altogether.

\vspace{0.05in}
\begin{table}[!h!tbp]
\begin{center}
\begin{minipage}{3.5 in}
\begin{ruledtabular}
\begin{tabular}{l||cccc}
\multicolumn{5}{c}{\hspace{0.4in}\underline{$W$+jets Matrix Method Normalization}}\vspace{0.1in} \\
                 &  1 jet  &  2 jets &  3 jets &  4 jets  \\
\hline
Electron channel &   1.39  &   1.42  &   1.14  &   0.88   \\
Muon channel     &   1.55  &   1.66  &   1.50  &   1.16
\end{tabular}
\end{ruledtabular}
\vspace{-0.1 in}
\caption[mmwjetsfactors]{Scale factors applied to the LO $W$+jets
samples, to match the data from the matrix method normalization.}
\label{mmwjets-factors}
\end{minipage}
\end{center}
\end{table}

Once the overall normalization to data is performed, we still need to
correct the relative composition of $Wb\bar{b}$ and $Wc\bar{c}$ with
respect to $Wjj$. We assume the same ratio applies for $Wb\bar{b}$ and
$Wc\bar{c}$. We have used our data to measure this heavy flavor ratio
in the form of a scale factor $\alpha$ such that:
$$
\alpha(Wb\bar{b}+Wc\bar{c})+Wjj+\ttbar+{\rm QCD = Data,}
$$ where $Wb\bar{b}$ and $Wc\bar{c}$ are the yields as given by the
{\alpgen} cross section. However, we should not normalize the
background to the data in the signal region, so we have chosen a
factor 1.5, and assigned a 30\% uncertainty to it, based on the
results shown in Table~\ref{alpha-factors} for the zero-tag case. The
one-jet bin can also be utilized to derive $\alpha$ but the result is
more difficult to generalize to the signal region. The zero-tag sample
is not used for the measurement, has ample statistics, and shows very
similar sensitivity to the heavy flavor ratio as the signal
region. The zero-tag sample contains about 10\% (20\%) $Wc\bar{c}$ in
the 1jet (2jet) bin. The $Wb\bar{b}$ fraction here is much
smaller. But the sum of $Wc\bar{c} + Wb\bar{b}$ is large enough to be
able to measure the relative heavy flavor scale factor.

\clearpage

\begin{table}[!h!tbp]
\begin{center}
\begin{minipage}{4.5 in}
\begin{ruledtabular}
\begin{tabular}{l||cccc}
\multicolumn{5}{c} {\hspace{0.5in}\underline{Scale Factor $\alpha$ to Match Heavy Flavor Fraction to Data}}\vspace{0.1in} \\
                 &     1 jet      &      2 jets     &      3 jets     &       4 jets     \\
\hline
% these numbers are Yann's alphas, but multiplied by 1.5, so they should be close to 1.5:
Electron Channel  &                 &                 &                 &                  \\
~~0 tags          & 1.53 $\pm$ 0.10 & 1.48 $\pm$ 0.10 & 1.50 $\pm$ 0.20 &  1.72 $\pm$ 0.40 \\
~~1 tag           & 1.29 $\pm$ 0.10 & 1.58 $\pm$ 0.10 & 1.40 $\pm$ 0.20 &  0.69 $\pm$ 0.60 \\
~~2 tags          &       ---       & 1.71 $\pm$ 0.40 & 2.92 $\pm$ 1.20 & -2.91 $\pm$ 3.50 \\
Muon Channel      &                 &                 &                 &                  \\
~~0 tags          & 1.54 $\pm$ 0.10 & 1.50 $\pm$ 0.10 & 1.52 $\pm$ 0.10 &  1.38 $\pm$ 0.20 \\
~~1 tag           & 1.11 $\pm$ 0.10 & 1.52 $\pm$ 0.10 & 1.32 $\pm$ 0.20 &  1.86 $\pm$ 0.50 \\
~~2 tags          &       ---       & 1.40 $\pm$ 0.40 & 2.46 $\pm$ 0.90 &  3.78 $\pm$ 2.80
% These numbers are the alphas (when 1.5 is applied), so they should be close to 1
%$\mu$+jets 0 tag & 1.03 $\pm$ 0.1 & 1.00 $\pm$ 0.1&  1.01 $\pm$ 0.1 & 0.92 $\pm$ 0.2 \\
%$\mu$+jets 1 tag & 0.74 $\pm$ 0.1 & 1.01 $\pm$ 0.1&  0.88 $\pm$ 0.2 & 1.24 $\pm$ 0.5 \\
%$\mu$+jets 2 tag & ---            & 0.93 $\pm$ 0.4&  1.64 $\pm$ 0.9 & 2.52 $\pm$ 2.8 \\ \hline
%$e$+jets 0 tag   & 1.02 $\pm$ 0.1 & 0.99 $\pm$ 0.1&  1.00 $\pm$ 0.2 & 1.15 $\pm$ 0.4 \\
%$e$+jets 1 tag   & 0.86 $\pm$ 0.1 & 1.05 $\pm$ 0.1&  0.93 $\pm$ 0.2 & 0.46 $\pm$ 0.6 \\
%$e$+jets 2 tag   & ---            & 1.14 $\pm$ 0.4&  1.95 $\pm$ 1.2 &-1.94 $\pm$ 3.5 \\
\end{tabular}
\end{ruledtabular}
\vspace{-0.1 in}
\caption[alphafactors]{Scale factor $\alpha$ for the $Wb\bar{b}$ and
$Wc\bar{c}$ yields to match the data in each jet bin, for 0 tag, 1
tag, and 2 tag samples. The uncertainties are statistical only.}
\label{alpha-factors}
\end{minipage}
\end{center}
\end{table}

% (The following may well be true, but let's not promise it
% publically in case we do not have time to do it!)
%
% We intend to cross check this scaling and its uncertainty by
% measuring the same $\alpha$ factor in our signal region using a
% sample with low discriminant output value from the multivariate
% analysis that will not be used in the final measurement.

Figure~\ref{fig:alphaseqzerotag} shows graphically the values of
$\alpha$ in the zero tag sample for electrons and muons in each jet
bin (eight numbers in total).

\begin{figure}[!h!tbp]
\includegraphics[width=0.6\textwidth]
{figures/alpha-from-eqzerotag-sample}
\vspace{-0.1in}
\begin{minipage}{4in}
\caption[tag]{The values of $\alpha$ in the zero tag sample for
electron and muon channel in each jet multiplicity. The eight values
can be read in Table.~\ref{alpha-factors}.}
\label{fig:alphaseqzerotag}
\end{minipage}
\end{figure}

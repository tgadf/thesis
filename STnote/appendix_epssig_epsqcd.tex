%---------------------------------------------------------------------
%  Writeup of the improved search of Run II data for single top quark
%  production at DZero.
%  Started: Oct 2006
%  Authors: The Single Top Working Group
%---------------------------------------------------------------------
%

\appendix
\section*{Appendix 2 --- Lepton Isolation Probabilities}
\label{epsilons}

Normalization of the $W$+jets and multijet backgrounds to data is
performed using the matrix method as explained in
Section~\ref{matrix-method}. To determine the separate contributions
from signal-like real-lepton events and background fake-lepton events
in our data, we rely on the measurement of two probabilities,
$\varepsilon_{{\rm real-}\ell}$ and $\varepsilon_{{\rm fake-}\ell}$.

$\varepsilon_{{\rm real-}\ell}$ is the probability for a real lepton
to pass the isolation requirements. It is obtained using a $Z{\rar}
ee$ sample for the electrons and $Z {\rar} \mu\mu$ sample for the
muons. One of the leptons is ``tagged'' as \emph{tight}, then the
$\varepsilon_{{\rm real-}\ell}$ is calculated measuring the
probability of the other ``probe'' lepton to pass the
\emph{tight} selection cut.

The $\pt$ and $\eta$ dependence for the electron case is shown in
Fig.~\ref{cc-eps-signal}. The $\pt$ and jet multiplicity dependence in
the muon case is shown in Fig.~\ref{mu-eps-signal}. The average values
for $\varepsilon_{{\rm fake-}e}$ and $\varepsilon_{{\rm fake-}\mu}$ as
a function of the jet multiplicity are given in
Table~\ref{eps-signal}. The overall uncertainty for $\varepsilon_{{\rm
fake-}\ell}$ is $1\%$ statistical and $2\%$ systematic.

\vspace{-0.1in}
\begin{figure}[!h!tbp]
\includegraphics[width=0.34\textwidth]
{figures/eps_sig_el_pt.eps}
\hspace{0.5in}
\includegraphics[width=0.34\textwidth]
{figures/eps_sig_el_eta.eps}
\vspace{-0.1in}
\caption[ccepssignal]{Transverse momentum and detector
pseudorapidity dependence of $\varepsilon_{{\rm real-}e}$.}
\label{cc-eps-signal}
\end{figure} 

\vspace{-0.3in}
\begin{figure}[!h!tbp]
\includegraphics[width=0.36\textwidth]
{figures/eps_sig_mu_pt.eps}
\hspace{0.5in}
\includegraphics[width=0.36\textwidth]
{figures/eps_sig_mu_njets.eps}
\vspace{-0.1in}
\caption[muepssignal]{Transverse momentum and jet
multiplicity dependence of $\varepsilon_{{\rm real-}\mu}$.}
\label{mu-eps-signal}
\end{figure}

\vspace{-0.2in}
\begin{table}[!h!tbp]
\begin{center}
\begin{minipage}{2.5 in}
\begin{ruledtabular}
\begin{tabular}{c||cc}
\multicolumn{3}{c}{\underline{Real-Lepton Probabilities}}
\vspace{0.1in}\\
No. of jets & $\varepsilon_{{\rm real-}e}$
            & $\varepsilon_{{\rm real-}\mu}$ \\
\hline
  1  &  $(87.3 \pm 2.1)\%$  &  $(99.1 \pm 2.2)\%$  \\
  2  &  $(87.4 \pm 2.1)\%$  &  $(98.9 \pm 2.2)\%$  \\
  3  &  $(87.4 \pm 2.1)\%$  &  $(98.7 \pm 2.2)\%$  \\
  4  &  $(87.5 \pm 2.1)\%$  &  $(96.1 \pm 2.1)\%$
\end{tabular}
\end{ruledtabular}
\vspace{-0.1in}
\caption[epssignal]{Average values for $\varepsilon_{{\rm real-}\ell}$
in the electron and muon samples for different jet multiplicities.}
\label{eps-signal}
\end{minipage}
\end{center}
\end{table}

\vspace{-0.2in}
$\varepsilon_{{\rm fake-}\ell}$ is the probability for a fake lepton
to pass the isolation requirements. It is obtained using a data
sample. Assuming that the low $\met$ region ($\met < 10$~GeV) is
dominated by the multijet background, and that $\varepsilon_{{\rm
fake-}\ell}$ is independent of the $\met$ cut, then $\varepsilon_{{\rm
fake-}\ell}$ is determined as the ratio of \emph{tight} over
\emph{loose} events in the low $\met$ region.
We studied $\varepsilon_{{\rm fake-}e}$ as a function of several
parameters, finding no dependence on the lepton $\pt$ or lepton
$\eta$, but dependence on the trigger version (because the isolation
requirement in the trigger changed) and jet multiplicity. For
$\varepsilon_{{\rm fake-}\mu}$, we performed a similar set of studies
and found no need to parametrize as a function of the trigger version
as the muon isolation in the trigger did not change during this
period. However, we did paramterize as a function of the muon
pseudorapidity where there is a small dependence. Although there is
also dependence on muon transverse momentum, parametrizing in this as
well does not change the overall result, so we have not done this.
Figure~\ref{fig:eps-qcd-e} show the calculation of $\varepsilon_{{\rm
fake-}e}$ from the ratio of tight to loose events as a function of
{\met} for each jet multiplicity, with all trigger versions
combined. Figure~\ref{fig:eps-qcd-mu} shows $\varepsilon_{{\rm
fake-}\mu}$ as a function of the muon pseudorapidity. A summary of the
results is presented in Table~\ref{eps-qcd}.

\begin{figure}[!h!tbp]
\includegraphics[width=0.98\textwidth]
{figures/eps_qcd_el_alltrigs_met}
\vspace{-0.1in}
\caption[figepsqcde]{$\varepsilon_{{\rm fake}e}$ calculated in
the $0 < {\met} < 10$ GeV region of the distributions of tight/loose
events, with all trigger versions combined.}
\label{fig:eps-qcd-e}
\end{figure}

\begin{figure}[!h!tbp]
\includegraphics[width=0.98\textwidth]
{figures/eps_qcd_mu_eta}
\vspace{-0.1in}
\caption[figepsqcdmu]{$\varepsilon_{{\rm fake}\mu}$ as a function of
the muon pseudorapidity.}
\label{fig:eps-qcd-mu}
\end{figure}

\begin{table}[!h!tbp]
\begin{center}
\begin{minipage}{6.5 in}
\begin{ruledtabular}
\begin{tabular}{c||ccccc|c}
\multicolumn{7}{c}{\hspace{0.8in}\underline{Fake-Lepton Probabilities}}
\vspace{0.05in}\\
& \multicolumn{5}{c|}{ $\varepsilon_{{\rm fake-}e}$, Trigger Version} & \\
No. of jets & v8--v11   &     v12     &     v13a    &     v13b    &     v14     &     $\varepsilon_{{\rm fake-}\mu}$ \\
\hline 
 1 & $(11.2 \pm 0.5)\%$ & $(17.9 \pm 0.6)\%$ & $(18.7 \pm 1.2)\%$ & $(19.1 \pm 0.6)\%$ & $(18.5 \pm 0.6)\%$ & $40.8\%$ \\
 2 & $(12.8 \pm 1.0)\%$ & $(19.2 \pm 1.0)\%$ & $(18.8 \pm 2.2)\%$ & $(19.4 \pm 1.1)\%$ & $(22.0 \pm 1.2)\%$ & $35.8\%$ \\
 3 & $(13.6 \pm 1.5)\%$ & $(19.5 \pm 1.6)\%$ & $(19.8 \pm 3.4)\%$ & $(19.2 \pm 1.6)\%$ & $(19.4 \pm 1.7)\%$ & $34.2\%$ \\
 4 & $(10.0 \pm 2.8)\%$ & $(15.5 \pm 2.9)\%$ & $(20.9 \pm 8.6)\%$ & $(17.7 \pm 3.3)\%$ & $(20.7 \pm 3.7)\%$ & $30.9\%$
\end{tabular}
\end{ruledtabular}
\vspace{-0.1in}
\caption[epsqcd]{$\varepsilon_{{\rm fake-}e}$ as a function of
the trigger version and jet multiplicity, and $\varepsilon_{{\rm
fake-}\mu}$ averaged over $\eta$.}
\label{eps-qcd}
\end{minipage}
\end{center}
\end{table}

\clearpage
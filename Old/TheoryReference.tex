 


\section{Standard Model}
\begin{itemize}
\item All fundamental particles and their interactions are described by quantum field theory known as the Standard Model.
\item The Standard Model is a gauge theory based on the symmetry group \mbox{SU$(3)_{c}$ $\otimes$ SU$(2)_{L}$ $\otimes$ U$(1)_{Y}$}.
\item This group structure is broken by the vacuum, which triggers the spontaneous breaking of the SU$(2)_{L}$ $\otimes$ U$(1)_{Y}$ group. It is believed that the Higgs mechanism is responsible for the spontaneous symmetry breaking. This mechanism generates masses for the weak gauge bosons (W and Z) as well as generating mass for all fermions. To achieve the symmetry breaking a neutral scalar boson must be introduced into the theory. This particle is called the Higgs boson and to date has not yet been observed.
\item This model describes the interactions of the strong, weak, and electromagnetic forces.
\item There are two types of particles in the SM: fermions (particles with fractional) and bosons (particles with integer spin).
\item Within the fermions, there are two more fundamental categories: quarks and leptons.
\item The first generation of the quarks and leptons are stable (do not decay) and makes up all the matter in the world. The second and third generations are not stable and will decay into the first generation.
\item Each quark generation comes in a doublet with one quark having a charge $+\frac{2}{3}$e and the other having charge $-\frac{1}{3}$e. The quarks with charge $+\frac{2}{3}$e are commonly called the up-type quarks and quarks with charge $-\frac{1}{3}$e are called the down-type quarks.
\item Each lepton generation also comes in a doublet, but this time one of the particles has unit charge and the other has no charge. The particles with unit charge are the electron, muon, and tau lepton. The particles with no charge are the associated neutrino for the previously listed leptons.
\item Particles with integer spin are the force carries within the SM framework. The four known spin 1 bosons in the SM: the photon, the gluon, and the W and Z particles. The photon is the carrier of the electromagnetic force which couples to particles with electromagnetic charge, such as the electron or the up quark. The gluon is the carrier of the strong force which couples particles with color charge which includes all the quarks as well as the gluon itself. The W and Z bosons are the carriers of the weak force which couples particles with weak charge, which is all quarks and leptons.
\item Within the SM framework, neutrinos required to have exactly zero mass; however recent experiments have shown that at least two the neutrinos must have a non-zero mass. There is no mechanism within the SM to give neutrinos mass and therefore this is evidence that the SM of particle interactions is not a complete theory. Even though the SM is not the full underlying theory of all particles and their interactions, it has been an extremely successful tool used to compute cross sections for particle collisions which is essential in high energy physics.
\item There are three types of interactions described in the SM: the strong interaction between objects with color such as quarks and gluons, the weak interaction between particles with weak charge, and electromagnetic interaction between charged particles.
\item The weak interaction is mediated by the charged W boson and the neutral Z boson. Interactions involving the Z boson are called neutral current interactions and only occur between members of the same family or generation. In contrast, interactions involving the charged W boson can  occur between members of the same generation, but also between generations. One interesting feature of the Standard Model Lagrangian is that the mass matrix and the weak matrix are can not be simultaneously diagnolized for weakly interacting particles. This mean that the weakly interacting fermions described in the Table.~\ref{fermions} are not the same as the fermions that interact with the W boson. The relationship between the weak and mass eigenstates of the Standard Model Lagrangian is summarized by the Cabbibo-Kobayahski-Maskawa (CKM) unitary quark mixing matrix. The parameters in the CKM matrix can not calculated from first principles and must be determined from experiment.
\item The mass or strong eigenstates can be rotated into the weak eigenstates by the unitary CKM mixing matrix as shown in Eq.~\ref{ckm}.
\item The magnitudes of the CKM matrix elements as determined from experiment can be found in Eq.~\ref{ckmval}.
\item The strong interaction is mediated by 8 massless and neutral gauge bosons called gluons. Gluons interact with any particle that carries strong charge called color. In the Standard Model only quarks and gluons carry color charge. In the Standard Model there are three different color commonly called red (R), blue (B), and green (G). One interesting property of the strong interaction at low energies is that bare quarks can not travel long distances when they carry color charge. Instead, colored quarks combine with other quarks to form colorless bound states called mesons (q$\bar{q}^{'}$) or baryons (qqq) depending on the number of constituent quarks. This phenomenon is called comfinement. Examples of mesons are pions (up and down quarks) and kaons (up and strange quarks) and examples of baryons are the proton (two up quarks and down quark) and the neutron (two down and one up quark). At higher energies the strong interaction becomes relatively weak and quarks are allowed to break their confined states and travel as a bare color charge. As the quark begins to propagate, it polarizes the vacuum until enough energy is built up to create a quark-antiquark pair. This process, called hadronization, can repeat itself many times and will produce many new strongly interacting particles. The effect of hadronization is the production of a large number of hadrons in the same direction as the originating colored quark or gluon. The top quark, discussed in the next section, is the only quark that does not undergo hadronization due its extremely short lifetime.
\end{itemize}




The electromagnetic (EM) interaction arises from the requirement that the EM Lagrangian is invariant under an arbitrary phase rotation (\mbox{$\Psi^{'}(x) \rightarrow e^{i\alpha\Theta(x)}\Psi(x)$}) of any particle wave function. This requirement means that EM Lagrangian obeys a U(1) local gauge symmetry. To make the Lagrangian invariant under this rotation, a new gauge field is added to the EM Lagrangian. This new field is interpreted as the photon field and mediates interactions between particles with electromagnetic charge with strength $\alpha$. The weak interaction arises in a similar fashion except that the arbitrary phase, $\Theta(x)$, is replace with a vector, $\vec{\sigma}(x)$. The requirement that the weak Lagrangian is invariant means that it obeys SU(2) local gauge symmetry. Similarly to the U(1) case, the weak Lagrangian is modified by the addition of the new gauge bosons, $\vec{W}$, to preserve the SU(2) symmetry. The weak interactions are special in that they have been observed to maximally violate parity and thus are constructed in the Standard Model to only interact with left handed particles and right handed anti-particles.

The electromagnetic and weak interactions can be unified into a \mbox{SU(2)$_{L}$ $\otimes$ U(1)$_{Y}$} gauge group, where the subscript, L, refers to the left handed nature of the weak interactions and the subscript, Y, is called the weak hypercharge and is related to the electric charge. This unification produces the correct interaction structure of the electromagnetic and weak forces, however it fails to give masses to the gauge bosons of the weak interactions, the W and Z bosons, as well as the quarks and leptons. Therefore, the \mbox{SU(2)$_{L}$ $\otimes$ U(1)$_{Y}$} gauge group is said to be spontaneously broken. One possible solution to break electroweak symmetry is the Higgs mechanism. This mechanism generates masses for all quarks and leptons as well as the W and Z bosons by requiring a new scalar field, the Higgs field, in the Standard Model Lagrangian. To date, the Higgs particle has not been observed in any experiment designed for it's detection. 

The strength of the particles interaction with the Higgs field determ

Finally, the strong force arises from the requirement that strongly interacting particles, such as quarks, be invariant under a rotation by a unitary matrix (\mbox{$\vec{q_{f}}(x) \rightarrow e^{i\lambda_{\alpha}\Theta^{\alpha}(x)}\vec{q_{f}}(x)$}), where $\lambda_{\alpha}$ is one of the Gellman matrices. This requirement means that the strong interaction obeys SU(3) local gauge symmetry.



\subsection{Higgs Mechanism and Particle Masses}
The Standard Model is a gauge theory based on the symmetry group \mbox{SU$(3)_{c}$ $\otimes$ SU$(2)_{L}$ $\otimes$ U$(1)_{Y}$}. This group structure is broken by the vacuum, which triggers the spontaneous breaking of the SU$(2)_{L}$ $\otimes$ U$(1)_{Y}$ group. It is believed that the Higgs mechanism is responsible for the spontaneous symmetry breaking. This mechanism generates masses for the weak gauge bosons (W and Z) as well as generating mass for all fermions. To achieve the symmetry breaking a neutral scalar boson must be introduced into the theory. This particle is called the Higgs boson and to date has not yet been observed.

In the Standard Model all particle interactions are mediated by spin=1 guage bosons. There are three types of particle interactions described in this model that are mediated by four gauge bosons: the photon, the gluon, and the W and Z particles. The photon is the carrier of the electromagnetic force which couples to particles with electromagnetic charge, such as the electron or the up quark. The gluon is the carrier of the strong force which couples particles with color charge which includes all the quarks as well as the gluon itself. The W and Z bosons are the carriers of the weak force which couples particles with weak charge, which includes all quarks and leptons. A summary of the gauge bosons can be found in Table ~\ref{bosons}


\begin{table}[!h!tbp]
\begin{center}
\begin{tabular}{c|c|c|c}
\multicolumn{4}{c}
{\underline{Summary of Fermion Gauge Interactions}} \\
$1^{\rm{st}}$ Generation		&	$2^{\rm{nd}}$ Generation		&	$3^{\rm{rd}}$ Generation		&	Gauge Interactions	\\
\hline
Up (u)					&	Charm (c)					&	Top (t)					&	Strong, Weak, Electromagnetic	\\
Down (d)					&	Strange (s)				&	Bottom (b)					&	Strong, Weak, Electromagnetic	\\
Electron ($e$)				&	Muon ($\mu$)				&	Tau ($\tau$)				&	Weak, Electromagnetic	\\
Electron neutrino ($\nu_{e}$)	&	Muon neutrino ($\nu_{\mu}$)	&	Tau neutrino ($\nu_{\tau}$)	&	Weak	\\
\end{tabular}
\vspace{-0.1 in}
\caption[interactions]{Summary of Fermion Gauge Interactions}
\label{interactions}
\end{center}
\end{table}




One interesting feature of the Standard Model Lagrangian is that the mass matrix and the weak matrix are can not be simultaneously diagnolized for weakly interacting particles. This mean that the weakly interacting fermions described in the Table.~\ref{fermions} are not the same as the fermions that interact with the W boson. The relationship between the weak and mass eigenstates of the Standard Model Lagrangian is summarized by the Cabbibo-Kobayahski-Maskawa (CKM) unitary quark mixing matrix. The parameters in the CKM matrix can not calculated from first principles and must be determined from experiment.

The mass or strong eigenstates can be rotated into the weak eigenstates by the unitary CKM mixing matrix as shown in Eq.~\ref{ckm}.



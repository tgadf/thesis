 
\begin{itemize}
\item Crockroft-Walton: negatively charged hydrogen ions are produced in a magnetron surface plasma source. Final acceleration by electrostatic separators is 750 keV.
\item Linac: The linac is 150 (500 from Andy's thesis) long set of RF cavities that accelerates the H- ions up to 400 MeV. After the linac, the ions are set through a carbon foil to strip the two electrons leaving the bare positively charged proton. At this stage the protons are grouped into bunches. The bunches come from the fact that only protons with a certain momentum will receive a positive acceleration and others will receive a negative acceleration. 
\item Booster: This is a synchrotron accelerator, which accelerates the protons to 8 GeV. The booster is 500 feet in diameter. 
\item Main Injector: This is a synchrotron accelerator, which takes proton bunches from the booster and accelerates them to 150 GeV. The main injector was added in 2001.
\item $\bar{p}$ Production: 120 GeV protons from the main injector are used to create anti protons by colliding with a nickel target. About 20 $\bar{p}$s are produced for every million protons used. The protons are focused using lithium lens (figure this out). To reduce the proton emittance, anti-protons undergo a process of "stochastic cooling." The next step is the debuncher which reduces the momentum spread at the expense of elongated bunches. After the debuncher, the protons are sent to the accumulator, which collects enough protons to be sent to the main injector for acceleration to 150 GeV.
\item Tevatron: The Tevatron uses superconducting magnetics to accelerate protons and anti-protons to an energy of 980 GeV $\rightarrow$ $\sqrt{s} = 1.96$~TeV. The main injector loads four bunches at a time into the Tevatron until there are 36 bunches in total at which time the 36 bunches are accelerated up to 980 GeV. Specs: bunches = 36x36, (RunI 6x6) ; $N_{p}$ = 2.6 x $10^{11}$ ; $N_{\bar{p}}$ = 3.5 x $10^{10}$ ; $\mathcal{L}$ = 200e30
\item Sources (general): http://www-bd.fnal.gov/public/chain.html [72] from Philip Perea's thesis.
\item Sources (cooling): . Mohl, G. Petrucci, L. Thorndahl and S. van der Meer, Phys. Rep. 58, 76 (1980).
\end{itemize}

SMT
\begin{itemize}
\item Silicon wafers arranged in various ways
\item 6 barrels that are 12 cm long, which are concentric to the beam pipe. Each barrel has 4 layers of double sided silicon wafers providing a possibility of 4 hits for a central track ($\eta <$~1.1). Layers 1 and 3 are 90 degrees stereo and Layers 2 and 4 are orientated 2 degrees stereo.
\item There are 12 F-disk that are perpendicular to the beam pipe. The F-disks are made of 12 double sided trapezoidally shaped silicon wafers.
\item There are 4 H-disks. The H-disks are located at the end of the silicon detector. Each disk has 24 wedges of double sided silicon.
\item The full SMT has nearly 793,000 channels.
\end{itemize}

CFT
\begin{itemize}
\item Scintillating fibers that produce light when charged particles ionize the scintillator material. The light travels through wave guides out of the detector to photon counters called VLPC (visible light photon counter). The photons measured on the VLPC produce electron-hole which yield an electric pulse due to a 6V bias voltage. The VLPC operate at liquid Helium temperatures to reduce nose.
\item There are 8 super layers of the fiber tracker. Each layer has two layers of fibers. The innermost layer is 20 cm away from the beam pipe and the outermost is 52 cm.
\item Each cylindrical layer has two sets of fibers. One that is parallel with the beam pipe and one that is rotated by 3 degrees with respect to the beam axis.
\item The fiber tracker measures r-$\phi$,  and z information from the intersection of two fibers.
\item There are 71,680 scintillating fibers.
\item Each fiber is 835 $\mu$m in diameter. Each fiber is composed by a scintillating fiber surrounded by a coating that has a high index of refraction to ensure total internal reflection as the light travels through the wave guide.
\item Each charged particle produces an average of 10 photons per fiber.
\end{itemize}

Calorimeter
\begin{itemize}
\item Alternating active region and absorber region. The EM calorimeter uses depleted uranium for the absorber and liquid argon as the active region. absorber plates are 3mm deep in the CC and 4mm deep in the EC.
\item The fine hadronic uses a uranium-niobium alloy. The plates are 6mm thick.
\item The coarse hadronic calorimeter uses copper (CC) and steel (EC). The plates are 46.5mm thick.

\item Particles hit the absorber material and produce a shower of lower energy particles. These lower energy particles, in turn, produce more lower energy particles as they hit the next absorber cell. The energy of these particles is sampled as they ionize the liquid argon in the active region. This procedure continues until all the energy has been deposited in the active region.

\item The active region is liquid argon held at a bias voltage, which allows the electron-hole pair produced by the ionizing particle to travel to the collector pads. The charge collected on the pads is then proportional to the amount of energy deposited in the active region.
\item The average drift time of the electron-hole pair inside the liquid argon is $\sim$430 ns.

\item The central calorimeter provides coverage up to $|\eta| <$~1. There are two end cap calorimeters which provide coverage out to $|\eta| <$~4. Between the central and forward calorimeters are sets of scintillating tiles ( no absorber material ), which provide some energy measurement between the two calorimeters.

\item The EM calorimeter has four layers with a granularity in $\delta\eta \times \delta\phi = 0.1 \times 0.1$. The third layer of the calorimeter has a granularity of $\delta\eta \times \delta\phi = 0.05 \times 0.05$ so as the better sample the EM shower at the average shower maximum. The EM calorimeter contains nearly 21 radiation lengths ($X_{0}$) of depleted uranium.
\item The fine hadronic calorimeter has three layers in the central calorimeter and four layers in the forward calorimeter. The granularity of these layers is $\delta\eta \times \delta\phi = 0.1 \times 0.1$.
\item The coarse hadronic has one layer with a granularity of $\delta\eta \times \delta\phi = 0.2 \times 0.2$. The two hadronic calorimeters provide nearly 7 additional radiation lengths.

\item The CC, and two EC calorimeters are encased in a cryostat which keeps the temperature at 90K.
\item Because each calorimeter is encased in separate cyrostats there is not complete calorimeter coverage from $0.8 < |\eta| < 1.4$. To solve this, single calorimeter cells without an absorber layer ( I need more information here) 
\end{itemize}


Shielding
\begin{itemize}
\item Shielding surrounded the beam pipe in the accelerator tunnel reduces beam halo effects. This shielding was installed during RunI.
\item New shielding was added for RunII. This shielding consists of iron as a hadronic and electromagnetic absorber, polyethylene for absorbing neutrons due to it's high hydrogen cotent, and lead for absorbing gamma rays.
\item This shield reduces the extra radiation deposited in the muon system by a factor of 50-100.

\end{itemize}


Muon
\begin{itemize}
\item Muons do not interact very much in the calorimeter because of their heavy mass ($106$ MeV, which is slightly more than 200 times the electron mass). The total power emitted by bremsstrahlung goes as $\frac{1}{m^{6}}$, thus electrons will lose their energy by bremsstrahlung much more quickly than muons.
\end{itemize}


%The muon system at $\dzero$ consists of three layers called A, B, and C. The A layer is located immediately beyond the calorimeter but before the toroid magnet, while the B and C layers are located outside the magnet. The presence of the magnet means a muon have a transverse momentum of at 3 GeV/c$^{2}$ to register a hit in the B and C layers. Requiring a confirmed hit in the outer layers removes the large background of muons from in-flight of pions and Kaons. 


Each layer of the detector contains one set of drift chambers and one set of scintillator counters. 

Muon system
\begin{itemize}
\item The muon system has three layers of scintillators and drift chambers used to detector muons coming from the hard scatter.
\item The first layer (A) is located immediately outside the calorimeter.
\item Beyond the A layer is a 2,000 ton 1.9 Telsa toroid magnet which is used to bend the charged muons.
\item The second and third layers (B and C) are then located outside the toroid magnet. A muon must have at least $p_{T} > 3$~GeV to register a hit in the B and C layers due to the interactions induced by dense iron in the magnet. This reduces the background of muons from $\pi$/K decays.
\item The central muon detector provides coverage up to $|\eta| <$~1 and the forward detector provides coverage from  1~$< |\eta| <$2. The central detector has two sets of scintillators (one inside and one outside) and the forward region has three (1 inside the iron and two outside).
\item The scintillators are used for fast triggering decisions at L1. The scintillators also provide fast timing information to reject muons which do not originate from the interaction region. Muons from the interaction region will arrive within 10 ns after the bunch crossing.
\item The x-y-z position of a muon is measured by the wire chambers. The wire chambers are a series of wires held at a large voltage that are surrounded by a mixture of 84$\%$ Argon, $8\%$~$\rm{CF}_{4}$ and $8\%$~$\rm{CH}_{4}$. As the charged muon passes through the chamber it will ionize the gas which immediately begins to drift towards the wire. The position of the muon track is then determined by the current profile of the wire. Each wire has a typical resolution of $\sim$1 mm.
\item The wires are also used in fast trigger decisions.
\end{itemize}


Luminosity
\begin{itemize}
\item Question: How do you normalize your data? - Answer: Measure the rate for a well known process and compare.
\end{itemize}
\begin{itemize}
\item Measures the inelastic p$\bar{p}$ cross section.
\item Also makes a fast measurement of the z coordinate of the interaction vertex by the relative time difference of coincidence counts in the North and South detector. The z position of the interaction vertex is given by z = $\frac{c}{2}$($t_{-}$ - $t_{+}$), where $t_{\pm}$ is the time measured from the bunch crossing by the South (-) and North (+) detector. The time of flight resolution of the LM is 0.3 ns.
\item Each detector has 24 plastic scintillation counters with a PMT readout. The detectors are located at $\pm$~140 cm from the center of the dector along the beam axis. The detectors are located directly in front of the end cap calorimeters and provide. They provide an $\eta$ coverage of $2.7 < \eta < 4.4$.
%\item Luminosity is defined as $\mathcal{L} = \frac{f_{beam} \times \bar{N}_{LM}}{\sigma_{LM}} where $f_{beam}$ is the bunch crossing frequency, $\bar{N}_{LM}$ is the average number of inelastic collisions per bunch crossing, and $\sigma_{LM}$ is the effective inelastic cross section taking into account the acceptance and efficiency of the luminosity detector.
\end{itemize}


Trigger
\begin{itemize}
\item General: Bunch crossings happen at a rate of 1.7 MHz. We can only record roughly 50 events per second to tape. We must select the most interesting events with unique signatures.
\item Level1 is a hardware trigger elements designed to reduce the input rate to 1-2 kHz. At L1, the trigger knows about calorimeter tower ( $\delta\eta \times \delta\phi = 0.2 \times 0.2$ )$E_{T}$, hits in muon wires and scintillators, and tracks in the central fiber tracker. A decision must be made within 3.5 $\mu$s if the event is to be passed. The L1 accept rate is driven by the readout rate of the subdetector and deadtime associated with the readout.
\item Level2 uses detector specific processing boards and a global system board to make trigger decisions. The processes collect data from the L1 trigger system as well as readout information from the individual detectors. The processors use this information to form physics objects such as electrons, jets, and missing et. The global processor uses this information to make decisions based on event wide kinematics. Events passing the level2 trigger are readout fully and ready for the third and final trigger decision stage.
\item After a Level2 trigger accept, the information from each sub detector is readout and stored in memory on a single board computer located in the detector readout crate. Once the event fragment has been collected, it is sent to the Level3 computer farm. The first step of the farm is to combine event fragments from each subdetector to form a complete event. A trigger decision is then made on the full event kinematics using information from each subdetector. If the event satistfies the criteria of the level3 trigger the event is then stored to tape.
\end{itemize}


